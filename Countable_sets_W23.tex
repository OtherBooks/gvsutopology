\achapter{19}{Countable and Uncountable Sets}\label{chap:countable}


\vspace*{-17 pt}
\framebox{
\parbox{\dimexpr\linewidth-3\fboxsep-3\fboxrule}
{\begin{fqs}
\item What do we mean by the cardinality of a set?
\item What does it mean for a set to be countably infinite?
\item What does it mean for a set to be uncountable?

\item \end{fqs}}}% \hspace*{3 pt}}

\vspace*{13 pt}

\csection{Introduction}\label{sec_count_set_intro}

How big is a set? When a set is finite, we can count the number of elements in the set and answer the question directly. When a set is infinite, the question is a little more complicated. For example, how big is $\Z$? How big is $\Q$? Since $\Z$ is a subset of $\Q$, we might think that $\Q$ contains more elements than $\Z$. But $\Z$ is infinite and how many more elements can we have than infinity? We will answer that question in this section.

If two finite sets have the same number of elements, then it should seem natural to say that the sets are of the same size. How do we extend this to infinite sets? If two finite sets have the same number of elements, then we can pair each element in one set with exactly one element in the other. This is exactly what a bijection does. So we will say that two sets $A$ and $B$ (either finite or infinite) have the same ``size" if there is a bijection between the sets.

\begin{definition} Two sets $A$ and $B$ have the same \textbf{cardinality}\index{cardinality of a set} if there is a bijection $f: A \to B$. 
\end{definition}

We use the word \emph{cardinality} instead of number of elements because we can't actually count the number of elements in an infinite set. We denote the cardinality of the set $A$ by $|A|$. We can then define a relation on the collection of sets as follows: two sets $A$ and $B$ are related if $|A| = |B|$. In other words, set $A$ is related to set $B$, denoted $A \simeq B$, if there is a bijection $f : A \to B$. 

\begin{pa} \label{pa:countability} Let $A$, $B$, and $C$ be sets. 
\be
\item Show that $A \simeq A$.

\item Show that if $A \simeq B$, then $B \simeq A$.

\item Show that if $A \simeq B$ and $B \simeq C$, then $A \simeq C$. 

\ee

\end{pa}

\begin{comment}

\ActivitySolution

\be
\item The function $id : A \to A$ defined by $id(a) = a$ for all $a \in A$ is clearly a bijection, so $A \simeq A$ and $\simeq$ is a reflexive relation.

\item Assume that $A \simeq B$. Then there is a bijection $f : A \to B$. To show that $B \simeq A$, we need to find a bijection from $B$ to $A$. Define $g : B \to A$ by 
\[g(b) = a \text{ when } f(a) = b.\]
First we show that $g$ is actually a function. If $b \in B$ so that $g(b) = a_1$ and $g(b) = a_2$, then $b = f(a_1) = f(a_2)$. Since $f$ is an injection, we know $a_1=a_2$. Thus, $g$ assigns exactly one output for each input and $g$ is a function. Next we show that $g$ is an injection. Suppose $b_1, b_2 \in B$ with $g(b_1) = g(b_2)$. Since $f$ is a surjection, there exist $a_1, a_2 \in A$ with $f(a_1) = b_1$ and $f(a_2) = b_2$. So 
\[a_1 = g(b_1) = g(b_2) = a_2\]
and since $f$ is a function we know $a_1=a_2$ implies $f(a_1)=f)a_2)$. Thus, $b_1 = f(a_1) = f(a_2) = b_2$ and $g$ is an injection. 
To complete our proof that $g$ is a bijection, we show that $g$ is a surjection. Let $a \in A$. Then $f(a) \in B$. Let $b = f(a)$. By definition, $g(b)=a$ and $g$ is a  surjection. Therefore, $g : B \to A$ is a bijection and $B \simeq A$. Thus, the relation $\simeq$ is a symmetric relation.

\item Assume that $A \simeq B$ and $B \simeq C$. Then we can find bijections $f : A \to B$ and $g : B \to C$. We need to show that there is a bijection from $A$ to $C$. The natural candidate is the composite $g \circ f$. To show that $g \circ f$ is an injection, suppose $a_1, a_2 \in A$ with $(g \circ f)(a_1) = (g \circ f)(a_2)$. Then $g(f(a_1)) = g(f(a_2))$. Since $g$ is an injection, we know $f(a_1) = f(a_2)$. Since $f$ is an injection, we can conclude $a_1=a_2$. Therefore, $g \circ f$ is an injection. To show $g \circ f$ is a surjection, choose an arbitrary $c \in C$. Since $g$ is a surjection, there is a $b \in B$ so that $g(b) = c$. Since $f$ is a surjection, there is an $a \in A$ so that $f(a)=b$. Then $(g \circ f)(a) = g(f(a)) = g(b) = c$ and $g \circ f$ is a surjection. Thus, $g \circ f : A \to C$ is a bijection and $A \simeq C$. 

\ee

\end{comment}


%\begin{theorem} The relation $\simeq$ is an equivalence relation.
%\end{theorem}

\csection{Countable Sets}\label{sec_count_set}

Preview Activity \ref{pa:countability} shows that the relation $\simeq$ is an equivalence relation. An equivalence class of the equivalence relation $\simeq$ is the collection of sets with the same cardinality. For example, $[\{1,2,3, \ldots, n\}]$ is the collection of sets with exactly $n$ elements, and there is one of these for each $n \in \N$. We call all of these sets \emph{finitely countable}. Note that $[\N]$ is disjoint from any $[\Z_n]$. We denote the cardinality of $\N$ by the symbol $\aleph_0$ (read ``aleph null" -- $\aleph$ is the first letter of the Hebrew alphabet). Any set in $[\N]$ is said to be \emph{infinitely countable}. A set is \emph{countable} if it is either finitely or infinitely countable. We have not yet seen an infinite set that is not in $[\N]$. Do any exist? In other words, is every set countable? We answer that question in the next sections. 

\csection{The Cardinality of the Set of Integers}\label{sec_card_set_int}
The set of natural numbers is a subset of the set of integers. Should we expect then that $|\Z|$ is greater than $|\N|$? In other words, is the set of integers really ``larger" that the set of natural numbers? The next theorem answers this question. 

\begin{theorem} The set $\Z$ of integers is countable.
\end{theorem}

\begin{proof} All we need do is illustrate a one-to-one correspondence between the elements of $\N$ and the integers. 
\begin{center}
\begin{tabular}{c c c c c c c} \\
$\N$ &1 &2 &3 &4 &5 &$\ldots$ \\
	&$\downarrow$ &$\downarrow$ &$\downarrow$ &$\downarrow$ &$\downarrow$ &$\ldots$ \\
$\Z$ &0 &1 &-1 &2 &-2 &$\ldots$
\end{tabular}
\end{center}
In other words, define $f : \N \to \Z$ by $f(n) = 
\begin{cases}
0, &\text{ if $n=1$}, \\
\frac{n}{2}, &\text{ if $n$ is even}, \\
-\frac{n-1}{2}, &\text{ if $n>1$ and $n$ is odd.}
\end{cases}$ \\

The table above shows that $f$ is a bijection. Therefore, $|\Z| = |\N|$. 
\end{proof}

So even though $\N$ is a proper subset of $\Z$, the two sets have the same cardinality. This idea can be used to actually define an infinite set.

\begin{definition} A set $A$ is \textbf{infinite}\index{infinite set} if there is a proper subset $B$ of $A$ and a bijection $f : A \to B$. 
\end{definition}

In other words, a set if infinite if it contains a proper subset with the same cardinality as the whole set. 

\csection{The Cardinality of the Set of Rational Numbers} \label{sec_card_set_rational}

Now we know $|\Z| = |\N|$, but $\Q$ is obtained from $\Z$ by including an infinite number of multiplicative inverses. So should we expect $|\Q| = |\Z| = |\N|$ or $|\Q| > |\Z|$? The next theorem answers this question. 

\begin{theorem} The set $\Q$ of rational numbers is countable.
\end{theorem}

\begin{proof} This proof is based on one in \emph{Fractals, Chaos, Power Laws}, by M. Schroeder, W.H.Freeman and Company, 1991. We first show that there is a one-to-one correspondence between the set $\Q^+$ of positive rationals and the set $\Z^+$. We will work with rational numbers $\frac{a}{b}$ with $a,b \in \Z^+$ and $\gcd(a,b)=1$. Define a function $K : \Q^+ \to \Z^+$ by 
\[K \left( \frac{a}{b} \right) = p_1^{2\alpha_1} p_2^{2\alpha_2} \cdots p_k^{2\alpha_k} q_1^{2\beta_1-1} q_2^{2\beta_2-1} \cdots q_m^{2\beta_m-1},\]
where $p_1^{\alpha_1} p_2^{\alpha_2} \cdots p_k^{\alpha_k}$ is the prime power decomposition of $a$ and $q_1^{\beta_1} q_2^{\beta_2} \cdots q_m^{\beta_m}$ is the prime power representation of $b$ (to avoid having to constantly consider 1 as a separate case, we call $1^1$ the prime power decomposition of 1). We claim that $K$ is a bijection. 

Suppose $K\left( \frac{a}{b} \right) = K\left( \frac{c}{d} \right)$ for some $a,b,c,d \in \Z^+$. Let $p_1^{\alpha_1} p_2^{\alpha_2} \cdots p_k^{\alpha_k}$ be the prime power decomposition of $a$, $q_1^{\beta_1} q_2^{\beta_2} \cdots q_m^{\beta_m}$ the prime power decomposition of $b$, $r_1^{\gamma_1} r_2^{\gamma_2} \cdots r_l^{\gamma_l}$ the prime power decomposition of $c$, and $s_1^{\delta_1} s_2^{\delta_2} \cdots s_n^{\delta_n}$ the prime power decomposition of $d$. Then 
\begin{equation} \label{Q_countable_1}
p_1^{2\alpha_1} p_2^{2\alpha_2} \cdots p_k^{2\alpha_k} q_1^{2\beta_1-1} q_2^{2\beta_2-1} \cdots q_m^{2\beta_m-1} = r_1^{2\gamma_1} r_2^{2\gamma_2} \cdots r_l^{2\gamma_l} s_1^{2\delta_1-1} s_2^{2\delta_2-1} \cdots s_n^{2\delta_n-1}.
\end{equation}
Since $\gcd(a,b) = \gcd(c,d)=1$, we know $p_i \neq q_j$ and $r_i \neq s_j$ for any $i,j$. For $1 \leq i \leq k$, we know $p_i$ appears an even number of times in the prime power factorization of $K\left( \frac{a}{b} \right)$. So $p_i$ must equal some prime in the prime power factorization of $K\left( \frac{c}{d} \right)$ and must appear an even number of times in this factorization. Since each $s_j$ appears an odd number of times in the prime power factorization of $K\left( \frac{c}{d} \right)$, it must be the case that, after suitable relabeling, $p_i = r_i$ and $2\alpha_i = 2\gamma_i$. Thus, $\alpha_i = \gamma_i$. If $k > l$, then $p_k$ will be a prime that appears an even number of times in the prime power factorization of $K\left( \frac{a}{b} \right)$, but with no corresponding prime appearing exactly an even number of times in the prime power factorization of $K\left( \frac{c}{d} \right)$. Therefore, $k \leq l$. Similarly, we have $l \leq k$, so $k=l$. Thus, $a = c$. Cancelling common factors from both sides of (\ref{Q_countable_1}) shows 
\begin{equation} \label{Q_countable_2}
q_1^{2\beta_1-1} q_2^{2\beta_2-1} \cdots q_m^{2\beta_m-1} = s_1^{2\delta_1-1} s_2^{2\delta_2-1} \cdots s_n^{2\delta_n-1}.
\end{equation}
The uniqueness of prime factorizations gives us $m=n$, $q_i=s_i$, and $2\beta_i-1 = 2\delta_i-1$ for each $i$ (after suitable relabeling). Therefore, $m=n, q_i=s_i$, and $\beta_i=\delta_i$ for each $i$. This shows $b=d$. Thus, $\frac{a}{b} = \frac{c}{d}$ and $K$ is an injection.

\begin{comment}
Now we need to show $K$ is a surjection. Let $c \in \Z^+$. If $c=1$, then $K\left( \frac{1}{1} \right) = c$. If $c > 1$, then $c$ has prime power factorization $r_1^{2\gamma_1} r_2^{2\gamma_2} \cdots r_l^{2\gamma_l}$ for distinct primes $r_1, r_2, \ldots r_l$, and positive integers $\gamma_1, \gamma_2, \ldots, \gamma_l$. Rearrange the primes $r_i$ so that $\gamma_1, \gamma_2, \ldots, \gamma_k$ are even and $\gamma_{k+1}, \gamma_{k+2}, \ldots, \gamma_{l}$ are odd. Define $p_1, p_2, \ldots, p_k$, $q_1, q_2, \ldots, q_m$, $\alpha_1, \alpha_2, \ldots, \alpha_k$, and $\beta_1, \beta_2, \ldots, \beta_m$ by 
\begin{itemize}
\item $p_1=r_1, p_2=r_2, \ldots, p_k=r_k$, 
\item $q_1 = r_{k+1}, q_2=r_{k+2}, \ldots, q_m=r_{l}$, 
\item $2\alpha_1 = \gamma_1, 2\alpha_2 = \gamma_2, \ldots, 2\alpha_k = \gamma_k$, and 
\item $2\beta_1-1 = \gamma_{k+1}, 2\beta_2-1 = \gamma_{k+2}, \ldots, 2\beta_m-1 = \gamma_{l}$. 
\end{itemize}
Then, if $a = p_1^{\alpha_1} p_2^{\alpha_2} \cdots p_k^{\alpha_k}$ and $b= q_1^{\beta_1} q_2^{\beta_2} \cdots q_m^{\beta_m}$, we have   
\[c = p_1^{2\alpha_1} p_2^{2\alpha_2} \cdots p_k^{2\alpha_k} q_1^{2\beta_1-1} q_2^{2\beta_2-1} \cdots q_m^{2\beta_m-1} = \frac{a}{b}.\]
Thus, $K$ is a surjection. Since $K$ is both an injection and surjection, we see that $K$ is a bijection.
\end{comment}

It is left to the reader to show that $K$ is a surjection and, consequently, a bijection. From here, we can easily see that the function $L : \Q \to \Z$ defined by $L(q) = 
\begin{cases}
K(q), &\text{ if $q>0$}, \\
0, 	&\text{ if $q=0$}, \\
-K(-q), &\text{ if $q<0$}
\end{cases}$
is a bijection. Thus, $|\Q| = |\Z| = |\N|$ and $\Q$ is a countable set. 
\end{proof}

\csection{The Cardinality of the Set of Real Numbers} \label{sec_card_set_real}

Now that we know $|\Q| = |\Z| = |\N|$ we might expect that every infinite set is countable. Surprisingly enough, that is not true as the next theorem shows. 

\begin{theorem} \label{thm:R_uncountable} The set of real numbers $\R$ is not countable.
\end{theorem}

\begin{proof} We proceed by contradiction and assume $\R$ is a countable set. Then there is a bijection $f : \N \to \R$. Since every real number has a finite, repeating, or infinite and non-repeating decimal expansion, let the decimal expansions of the real numbers be given by  
\begin{align*}
f(1) &= a_1.a_{1,1} a_{1,2} a_{1,3} a_{1,4} \ldots \\
f(2) &= a_2.a_{2,1} a_{2,2} a_{2,3} a_{2,4} \ldots \\
f(3) &= a_2.a_{3,1} a_{3,2} a_{3,3} a_{3,4} \ldots \\
f(4) &= a_2.a_{4,1} a_{4,2} a_{4,3} a_{4,4} \ldots \\
&  \vdots  \\
f(n) &= a_n.a_{n,1} a_{n,2} a_{n,3} a_{n,4} \ldots \\
&  \vdots 
\end{align*}
Now we will construct a real number that is not equal to $f(k)$ for any natural number $k$. Let 
\[x = 0.x_1 x_2 x_3 x_4 \ldots\]
where $x_i = 
\begin{cases}
2, &\text{ if $a_{i,i} \neq 2$}, \\
3, &\text{ if $a_{i,i} = 2$}.
\end{cases}$
Then, for each $i \in \N$, the decimal expansion of $x$ differs from the decimal expansion of $f(i)$ at the $i^{\text{th}}$ position. Thus, $x \neq f(i)$ for each $i \in \N$. Since $x$ is a real number and all of the real numbers can be written as $f(i)$ for some $i \in \N$, we have a contradiction. Therefore, $\R$ cannot be countable. 
\end{proof}

The argument used in the proof of Theorem \ref{thm:R_uncountable} is an example of what is referred to as Cantor's diagonalization argument. Theorem \ref{thm:R_uncountable} shows that $\R$ is in some sense a ``bigger" set than $\Q$, or $|\R| > |\Q|$. Another way to put this is that $|\R|$ is a ``larger" infinity than $\aleph_0$. So there are different ``sizes" of infinity. An infinite set that is not countable is called \emph{uncountable}. 

\csection{More Uncountable Sets}\label{sec_uncount_set}

Now that we have different sizes of infinity, how many different sizes can there be? Is there a set $S$ so that $|S| > |\R|$? To answer this question, we consider the power set of a set. 

\begin{definition} Let $A$ be a set. The \textbf{power set}\index{power set} $2^A$ is the set of all subsets of $A$. 
\end{definition}

In other words, $2^A= \{ S \subseteq A\}$. The next theorem is an important result due to Georg Ferdinand Ludwig Philipp Cantor (March 3, 1845, St. Petersburg, Russia � January 6, 1918, Halle, Germany), a major figure in the development of formal set theory. 

\begin{theorem}[Cantor's Theorem] Let $A$ be a set. Then $|2^A| > |A|$. 
\end{theorem}

\begin{proof} Let $A$ be an arbitrary set. We proceed by contradiction and assume $|2^A| = |A|$. Let $f : A \to 2^A$ be a bijection. So for each $a \in A$, the image $f(a)$ is a subset of $A$. Now it will be the case that $a$ is either an element of the set $f(a)$ or $a$ is not an element of the set $f(a)$. Let 
\[S = \{a \in A : a \notin f(a)\}.\]
Note that $S$ is a subset of $A$. That means $S = f(x)$ for some $x \in A$. Now, either $x \in f(x)$ or $x \notin f(x)$. If $x \in f(x)$, then $x \notin S$ by the definition of $S$. But if $x \notin S$, then $x \notin f(x)$, a contradiction. So it must be the case that $x \notin f(x)$. But then $x \in S$ by the definition of $S$. However, $x \in S$ means $x \in f(x)$, again a contradiction. So we have found a subset $S$ of $A$ that is not the image of any element of $A$ under $f$. This cannot happen if $f$ is a bijection. We conclude that no such bijection exists and $|2^A| \neq |A|$. 

Now we must show $|2^A| > |A|$. We do so by finding a subset $A'$ of $2^A$ for which $|A| = |A'|$. Let 
\[A' = \{\{a\} : a \in A\}.\]
In other words, $A'$ is the collection of single elements subsets of $A$. Clearly, the function $g : A \to A'$ defined by $g(a) = \{a\}$ is a bijection. Thus, $|A| = |A'|$. Since $|2^A| \geq |A'| = |A|$ and $|2^A| \neq |A|$, we can only conclude $|2^A| > |A|$. 
\end{proof}
 
If we begin with $|\N|$, we can then construct countably many different ``sizes" of infinity:
\[|\N| < |2^{\N}| < |2^{2^{\N}}| < |2^{2^{2^{\N}}}| < \cdots \]

So the cardinality of the different sizes of infinity is at least countably infinite. A good question to ask is whether there are any more. The answer is that we don't know. In fact, there is a famous unsolved conjecture in mathematics, the \emph{Continuum Hypothesis} that says there is no set $S$ so that $\aleph_0 < |S| < |\R|$. 

\csection{Summary}\label{sec_count_set_summ}
Important ideas that we discussed in this section include the following.

%Cocountable topology. The cocountable or countable complement topology on a set $X$ is the set consisting of the empty set and all subsets of $X$ whose complements are countable. This is simiilar to the cofinite topology. 

%A product of countable sets is countable. Any finite product of countable sets is countable.

% Every subset of a countable set is countable.

%Unions of countable sets are countable.

% The set of irrational numbers is uncountable.






