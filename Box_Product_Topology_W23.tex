\achapter{19}{Products of Topological Spaces}\label{chap:Product_topology}


\vspace*{-17 pt}
\framebox{
\parbox{\dimexpr\linewidth-3\fboxsep-3\fboxrule}
{\begin{fqs}
\item 

\end{fqs}}}% \hspace*{3 pt}}

\vspace*{13 pt}

\csection{Introduction}\label{sec_prod_top_intro}
If we have two topological spaces $(X, \tau_X)$ and $(Y , \tau_Y)$, we might wonder if we can make the set $X \times Y$ into a topological space. A natural approach might be to take as the open sets in $X \times Y$ the sets of the form $U \times V$ where $U \in \tau_X$  and $V \in \tau_Y$. 

\begin{pa} \label{PA:pd_top_1} Let $X = \{a,b,c\}$ with $\tau_X = \{\emptyset, \{a\}, \{b\}, \{a,b\}, \{a,c\}, X\}$, and let $Y = \{1,2\}$ with $\tau_Y = \{\emptyset, \{1\}, Y\}$. 
\be
\item Let 
\begin{equation} \label{eq:prod_basis}
\B = \{U \times V \mid U \in \tau_X \text{ and } V \in \tau_Y\}.
\end{equation}
List all of the sets in $\B$ along with their elements. 
 
\item Assume that all of the sets in $\B$ are open sets in $X \times Y$. Should the set $A = \{(a,1), (a,2), (b,1)\}$ be an open set in $X \times Y$? Is the set $A$ of the form $U \times V$ for some open sets $U$ in $X$ and $V$ in $Y$? Explain. Is $\B$ a topology on $X \times Y$? 

\item If $\B$ is not a topology on $X \times Y$, what is the smallest collection of sets would we need to add to $\B$ to make a topology on $X \times Y$? Explain your process. 

\ee

\end{pa}

\begin{comment}

\ActivitySolution

\be
\item  We take all products of open sets in $X$ and $Y$ to obtain the basis elements
\begin{align*}
\emptyset & \\
\{a\} \times \{1\} &= \{(a,1)\} \\
\{a\} \times Y &= \{(a,1), (a,2)\} \\
\{b\} \times \{1\} &= \{(b,1)\} \\
\{b\} \times Y &= \{(b,1), (b,2)\} \\
\{a,b\} \times \{1\} &= \{(a,1), (b,1)\} \\
\{a,b\} \times Y &= \{(a,1), (a,2), (b,1), (b,2)\} \\
\{a,c\} \times \{1\} &= \{(a,1), (c,1)\} \\
\{a,c\} \times Y &= &= \{(a,1), (a,2), (c,1), (c,2)\}
X \times \{1\} &= \{(a,1), (b,1), (c,1)\} \\
X \times Y &= \{(a,1), (a,2), (b,1), (b,2), (c,1), (c,2)\}.
\end{align*}

 
\item  We have $A = (\{a\} \times Y) \cup (\{b\} \times \{1\})$, so $A$ is a union of open sets in $X \times Y$ and should be open. But $A$ is not of the form $U \times V$ for some open sets $U$ in $X$ and $V$ in $Y$, as it is not in the list of those sets in part (a). 


\item Since $\CB$ is not closed under unions, $\B$ is not a topology on $X \times Y$. However, if we consider all of the unions and finite intersections of sets in $\B$ we create the topology 
\[\{\emptyset, \{(a,1)\}, \{(b,1)\}, \{(a,1), (a,2)\}, \{(b,1), (b,2)\}, \{(a,1), (b,1)\}, \{(a,1), (c,1)\}, \{(a,1), (b,1), (b,2)\}, \{(a,1), (a,2), (b,1)\}, \{(a,1), (a,2), (c,1)\},  \{(a,1), (b,1), (c,1)\}, \{(a,1), (a,2), (b,1), (c,1)\}, \{(a,1), (b,1), (b,2), (c,1)\}, \{(a,1), (a,2), (b,1), (b,2)\}, \{(a,1), (a,2), (b,1), (b,2), (c,1)\}, X \times Y\}\]
on $X \times Y$. In other words, we consider $\CB$ a basis for a topology of $X \times Y$. 

\ee

\end{comment}


\csection{The Box and Product Topologies} \label{sec_box_prod_top}

In our preview activity we learned that we cannot make a topology on a product $X \times Y$ of topological spaces $(X, \tau_X)$ and $(Y , \tau_Y)$ with just the sets of the form $U \times V$ where $U \in \tau_X$ and $V \in \tau_Y$ as the open sets since the collection of these sets is not closed under arbitrary unions. What we can do instead is consider these unions of all of the sets of the form $U \times V$, where $U$ is open in $X$ and $V$ is open in $Y$. In other words, consider these sets to be a basis for the topology on $X \times Y$. 

\begin{activity} Let $(X, \tau_X)$ and $(Y, \tau_Y)$ be topological spaces, and let $\B$ be as defined in (\ref{eq:prod_basis}). Prove that $\B$ is a basis for a topology on $X \times Y$.

\end{activity}

\begin{comment}

\ActivitySolution Since $X \in \tau_X$ and $Y \in \tau_Y$, every point in $X \times Y$ is in a set in $\B$. So $\B$ satisfies condition 1 of a basis. Suppose $(x,y) \in (U_1 \times V_1) \cap (U_2 \times V_2)$. Then $x \in U_1 \cap U_2$ and $y \in V_1 \cap V_2$, and 
\[(x,y) \in (U_1 \cap U_2) \times (V_1 \cap V_2) \subseteq (U_1 \times V_1) \cap (U_2 \times V_2).\]
So $\B$ is a basis for a topology on $X \times Y$. 

\end{comment}

The topology we have been considering can be extended to a product of any collection of topological spaces and is called the \emph{box} topology.

\begin{definition} Let $(X_{\alpha}, \tau_{\alpha})$ be a collection of topological spaces for $\alpha$ in some indexing set $I$. The \textbf{box topology}\index{box topology} on the product $\Pi_{\alpha \in I} X_{\alpha}$ is the topology with basis
\[\B = \left\{ \Pi_{\alpha \in I} U_{\alpha} \mid U_{\alpha} \in \tau_{\alpha}  \text{ for each } \alpha \in I}.\]
\end{definition}

So we can always make the product of topological spaces into a topological space using the box topology. There is another way we could consider making a topology on a product. 

Given topological spaces $(X_{\alpha}, \tau_{\alpha})_{\alpha \in I}, for each $\beta \in I$ we define $\pi_{\beta}: \Pi_{\alpha \in I} X_{\alpha}  \to X_{\beta}$ by $\pi_{\beta}((x_{\alpha}_{\alpha \in I)) = x_{\beta}$. The function $\pi_{\beta}$ is the \emph{projection}\index{projection} of $\Pi_{\alpha \in I} X_{\alpha}$ onto $X_{\beta}$. We could then define 
\[\CS_{\beta} = \left\{ \pi_{beta}^{-1}(U_\beta} \mid U_{\beta} \text{ is open in } X_{\beta}\right\}.\]
We then let 
\[\CS = \bigcup_{\beta \in I} \CS_{\beta}.\]
Then $\CS$ generates a topology on $\Pi_{\alpha \in I} X_{\alpha}$. This topology is called the \emph{product topology}. 



These projection functions can help us determine when a function $f$ from a topological space $Y$ to $X_1 \times X_2$ is continuous.



This same construction works for any finite number of topological spaces, but the main ideas are contained in a product of two spaces, so we keep our focus on that.\footnote{It is possible to define a topology on an infinite collection of topological spaces, but we will not consider such a product here.}


\csection{Three Examples} \label{sec_prod_top_ex}

In this section we consider three specific examples of a product of topological spaces. 

\begin{activity} Let $X = [1,2]$ and $Y = [3,4]$ as subspaces of $\R^2$. 
\ba
\item Explain in detail what the product space $X \times Y$ looks like. 

\item Find, if possible, an open subset of $X \times Y$ that is not of the form $U \times V$ where $U$ is open in $X$ and $V$ is open in $Y$.

\ea

\end{activity}

\begin{comment}

\ActivitySolution

\ba
\item The product space $X \times Y$ contains all points of the form $(x,y)$ with $1 \leq x \leq 2$ and $3 \leq y \leq 4$ in $\R^2$. This set of points is the rectangle in $\R^2$ with vertices $(1,3)$, $(2,3)$, $(1,4)$ and $(2,4)$. 

\item The set $O = \{(x,y) \mid 1.1 < x < 1.3, 3.1 < y < 3.3\} \cup \{(x,y) \mid 1.7< x < 1.9, 3.7 < y < 3.9\}$ is equal to $(B(1.2,0.1) \times B(3.2,0.1)) \cup ((B(1.8,0.1) \times B(3.8,0.1))$ and so is open in $X \times Y$. However, $O$ is not of the form $U \times V$ where $U$ is open in $X$ and $V$ is open in $Y$. To see why, suppose to the contrary that $O = U \times V$ for some $U$ open in $X$ and $V$ open in $Y$. Since the points $(x,y)$ in $O$ only have $x \in (1.1,1.3) \cup (1.7,1.9)$ and $y \in (3.1,3.3) \cup (3.7,3.9)$, it follows that $U = (1,1,1.3) \cup (1.7,1.9)$ and $V = (3.1,3.3) \cup (3.7,3.9)$. But then $(1.8, 3.2)$ is in $U \times V$. Since $(1.8, 3.2)$ is not in $O$, we conclude that $O$ is not of the for $U \times V$ with $U$ open in $X$ and $V$ open in $Y$.  

\ea


\end{comment}



\begin{activity} Let $X = \R$ and $Y = S^1 = \{(x,y) \mid x^2 + y^2 = 1\}$, the unit circle as a subset of $\R^2$.  
\ba
\item Draw a picture of $\R$. For each $x \in \R$, the set $\R_x = \{(x, y) \mid y \in S^1\}$ is a subset of $\R \times S^1$. On your graph of $\R$, draw pictures of $\R_x$ for $x$ equal to $-1$, $0$, and $1$. Explain in detail what the product space $\R \times S^1$ looks like. 

\item Consider the sets of the form $B \cap S^1$, where $B$ is an open ball in $\R^2$ (relatively open sets in $S^1$). What do these sets look like?

\item Describe the shape of the basis elements for the product topology on $\R \times S^1$ that result from products of the form $U \times V$, where $U$ is an open interval in $\R$ and $V$ is the intersections of $S^1$ with an open ball in $\R^2$. 

\ea

\end{activity}

\begin{comment}

\ActivitySolution

\ba
\item We draw $\R$ as a horizontal line. At each point $x$ on the line, we draw a circle or radius 1 on the line as shown at left in Figure \ref{F:Product_example_tube} for $x=-1$, $x=0$, and $x=1$. The complete product space looks like an infinitely long tube as shown at right in Figure \ref{F:Product_example_tube} . 
\begin{figure}[h]
\begin{center}
\resizebox{!}{1.2in}{\includegraphics[trim=1.25cm 2.25cm 1.25cm 2.5cm, clip]{Product_example_tube_1.png}} \hspace{0.2in} \resizebox{!}{1.2in}{\includegraphics[trim=1.25cm 2.25cm 1.25cm 2.5cm, clip]{Product_example_tube_2.png}}
\caption{Left: $R_x$ for $x$ equal to $-1$, $0$, and $1$. Right: The product space.} 
\label{F:Product_example_tube}
\end{center}
\end{figure}
%\includegraphics[trim=left bottom right top, clip]{file}

\item When we intersect an open ball with a circle, we get an open arc on the circle. 

\item If we cross an open interval in $\R$ with an open arc on the circle, the resulting figure looks like a finite length trough as depicted in Figure \ref{F:Tube_open}. 
\begin{figure}[h]
\begin{center}
\resizebox{!}{1.2in}{\includegraphics[trim=1.25cm 2.25cm 1.25cm 3.0cm, clip]{Product_example_tube_3.png}} 
\caption{An example $U \times V$.} 
\label{F:Tube_open}
\end{center}
\end{figure}
%\includegraphics[trim=left bottom right top, clip]{file}
\ea

\end{comment}

\begin{activity} Let $2S^1 = \{(x,y) \mid x^2 + y^2 = 4\}$ be the unit circle as a subset of $\R^2$. In this activity we investigate the space $2S^1 \times  S^1$.  
\ba
\item Draw a picture of $2S^1$ in the $xy$-plane. For each $p \in \S^1$, the set $S^1_p = \{(p, y) \mid y \in S^1\}$ is a subset of $S^1 \times S^1$. On your graph of $S^1$, draw pictures of $S^1_p$ for $p$ equal to $(1,0)$, $\left(\frac{\sqrt{2}}{2}, \frac{\sqrt{2}}{2}\right)$, and $(0,1)$. Orient the graphs so that the copies of $S^1$ are perpendicular to $2S^1$. Explain in detail what the product space $2S^1 \times S^1$ looks like. 

\item Consider the sets of the form $B \cap S^1$, where $B$ is an open ball in $\R^2$. What do these sets look like?

\item Describe the shape of the basis elements for the product topology on $S^1 \times S^1$ that result from products of the form $U \times V$, where $U$ and $V$ are intersections of $S^1$ with open balls in $\R^2$. 

\ea

\end{activity}

\begin{comment}

\ActivitySolution

\ba
\item A picture of $2S^1$ and $S^1_p = \{(p, y) \mid y \in S^1\}$ for $p$ equal to $(1,0)$, $\left(\frac{\sqrt{2}}{2}, \frac{\sqrt{2}}{2}\right)$, and $(0,1)$ are shown at left in Figure \ref{F:Product_example_torus}. The product space $2S^1 \times S^1$ looks like a torus as illustrated at right in Figure \ref{F:Product_example_torus}.
\begin{figure}[h]
\begin{center}
\resizebox{!}{1.0in}{\includegraphics[trim=1.25cm 2.0cm 1.25cm 2.0cm, clip]{Product_example_torus_1.png}} \hspace{0.5in} \resizebox{!}{1.0in}{\includegraphics[trim=1.25cm 2.0cm 1.25cm 2.0cm, clip]{Product_example_torus_2.png}}
\caption{Left: Cross sections of the product space. Right: The product space. $2S^1 \times S^1$.} 
\label{F:Product_example_torus}
\end{center}
\end{figure}
%\includegraphics[trim=left bottom right top, clip]{file}

\item When we intersect an open ball with a circle, we get an open arc on the circle. 

\item We cross an open arc on $2S^1$ with an open arc on $S^1$ and the result is a slice of the torus both horizontally and vertically as illustrated in Figure \ref{F:Torus_open}.
\begin{figure}[h]
\begin{center}
\resizebox{!}{1.0in}{\includegraphics[trim=1.25cm 1.25cm 1.25cm 2.0cm, clip]{Product_example_torus_3.png}}
\caption{An example $U \times V$ on $2S^1 \times S^1$.} 
\label{F:Torus_open}
\end{center}
\end{figure}
\ea

\end{comment}

\csection{Products, Projections, and Continuous Functions on Products} \label{sec_prod_proj_cont}

Given topological spaces $(X_1, \tau_1)$ and $(X_2, \tau_2)$, we define $\pi_1: X_1 \times X_2 \to X_1$ and $\pi_2: X_1 \times X_2 \to X_2$  by $\pi_1((x,y)) = x$ and $\pi_2((x,y)) = y$. These functions $\pi_1$ and $\pi_2$ are the \emph{projections}\index{projection functions} of $X_1 \times X_2$ onto $X_1$ and $X_2$, respectively. These projection functions can help us determine when a function $f$ from a topological space $Y$ to $X_1 \times X_2$ is continuous.

\begin{activity} \label{act:projection_continuous} Let $(X_1, \tau_1)$ and $(X_2, \tau_2)$ be topological spaces and let $O_1$ be an open set in $X_1$. 
\ba
\item Determine which set is $\pi_1^{-1}(O_1)$. Verify your conjecture.

\item Explain why $\pi_1$ is continuous.

\ea

\end{activity}

\begin{comment}

\ActivitySolution

\ba
\item We will demonstrate that $\pi_1^{-1}(O_1) = O_1 \times X_2$. Let $(s,t) \in \pi_1^{-1}(O_1)$. Then $\pi_1((s,t)) = s \in O_1$. So $(s,t) \in O_1 \times X_2$ and $\pi_1^{-1}(O_1) \subseteq O_1 \times X_2$. Now let $(s,t) \in O_1 \times X_2$. Then $\pi_1((s,t)) = s \in O_1$ and $(s,t) \in \pi_1^{-1}(O_1)$. 

\item Recall that the sets $O_1 \times O_2$ with $O_1 \in \tau_1$ and $O_2 \in \tau_2$ form a basis for the product topology. Part (a) shows that the inverse image of an open set in $X_1$ under $\pi_1$ is open in $X_1 \times X_2$, so we conclude that $\pi_1$ is a continuous function.

\ea

\end{comment}

The same argument as in Activity \ref{act:projection_continuous} shows that $\pi_2$ is also a continuous function. Now suppose that $X_1$, $X_2$, and $Y$ are topological spaces, and that $f: Y \to X_1 \times X_2$ is a function. Then $\pi_1 \circ f$ maps $Y$ to $X_1$ and $\pi_2 \circ f$ maps $Y$ to $X_2$. Since the composition of continuous functions is continuous, we can see that if $f$ is continuous so are $\pi_1\circ f$ and $\pi_2 \circ f$. To determine if $f$ is a continuous function, it would be useful to know if the converse is true.

\begin{activity} Let $X_1$, $X_2$, and $Y$ be topological spaces, and let $f: Y \to X_1 \times X_2$ be a function.  Assume that both $\pi_1\circ f$ and $\pi_2 \circ f$ are continuous. Let $O_1$ be an open set in $X_1$ and $O_2$ an open set in $X_2$. 
\ba
\item What set is $(\pi_1\circ f)^{-1}(O_1)$? Verify your conjecture.

\item What set is $(\pi_2\circ f)^{-1}(O_2)$? Verify your conjecture.

\item Explain why $f$ is continuous.


\ea

\end{activity}

 
\csection{Properties of Products of Topological Spaces} \label{sec_prop_prod_top}

It is natural to ask what topological properties of the topological spaces $(X, \tau_X)$ and $(Y, \tau_Y)$ are inherited by the product $X \times Y$. We have studied Hausdorff, connected, and compact spaces, and we now consider those properties. 

\begin{activity} Let $(X, \tau_X)$ and $(Y, \tau_Y)$ be Hausdorff spaces.
\ba
\item What will it take to prove that the space $X \times Y$ with the product topology is Hausdorff?

\item Suppose that $(x_1,y_1), (x_2, y_2) \in X \times Y$. What does the fact that $X$ is Hausdorff tell us about $x_1$ and $x_2$? What can we say about $y_1$ and $y_2$?

\item Complete the proof of the following theorem.

\begin{theorem} If $(X, \tau_X)$ and $(Y, \tau_Y)$ are Hausdorff spaces, then $X \times Y$ with the product topology is a Hausdorff space. 
\end{theorem}

\ea

\end{activity}

\begin{comment}

\ActivitySolution

\ba
\item We need to show that if $a = (x_1,y_1)$ and $b = (x_2,y_2)$ are distinct points in $X \times Y$, then there are disjoint open sets $U$ and $V$ in $X \times Y$ such that $a \in U$ and $b \in V$. 

\item The fact that $X$ is Hausdorff tells us that there exist open sets $U_1$ and $U_2$ in $X$ such that $x_1 \in U_1$, $x_2 \in U_2$, and $U_1 \cap U_2 = \emptyset$. Similarly, the fact that $Y$ is Hausdorff tells us that there exist open sets $V_1$ and $V_2$ in $Y$ such that $y_1 \in V_1$, $y_2 \in V_2$, and $V_1 \cap V_2 = \emptyset$.

\item From part (b) we know that $(x_1,y_1) \in U_1 \times V_1$ and $(x_2, y_2) \in U_2 \times V_2$, and 
\[(U_1 \times V_1) \cap (U_2 \times V_2) = (U_1 \cap U_2) \times (V_1 \cap V_2) = \emptyset.\]
So the sets $U_1 \times V_1$ and $U_2 \times V_2$ separate $(x_1,y_1)$ and $(x_2, y_2)$ and $X \times Y$ is a Hausdorff space. 

\ea

\end{comment}

The proofs that a product of connected spaces is connected and that a product of compact spaces is compact are a bit more complicated. To prove that a product of two connected spaces is connected, we will use the following lemma.

\begin{lemma} \label{lem:Connected_union} Let $X$ be a topological space, and let $A_{\alpha}$ be a connected subset of $X$ for all $\alpha$ in some indexing set $I$. If $\bigcap_{\alpha \in I} A_{\alpha} \neq \emptyset$, then $\bigcup_{\alpha \in I} A_{\alpha}$ is connected. 
\end{lemma}

\begin{proof} Let $X$ be a topological space, and let $A_{\alpha}$ be a connected subset of $X$ for all $\alpha$ in some indexing set $I$. Assume $\bigcap_{\alpha \in I} A_{\alpha} \neq \emptyset$. To prove that $A = \bigcup_{\alpha \in I} A_{\alpha}$ is connected, assume that the pair of sets $U$ and $V$ satisfy
\[A \subseteq U \cup V, \ \ U \cap A \neq \emptyset, \ \ \text{ and  } \ \  U \cap V \cap A = \emptyset.\]
We will show that $V \cap A = \emptyset$, from which it will follow that there is no separation of $A$ and $A$ is connected. Since $U \cap A \neq \emptyset$, we know that $U \cap A_{\beta} \neq \emptyset$ for some $\beta \in I$. Since $A_{\beta}$ is connected, we know that $A_{\beta} \subseteq U$. Let $\gamma \in I$. The fact that $\bigcap_{\alpha \in I} A_{\alpha} \neq \emptyset$ means that there is an element in $A_{\gamma} \cap A_{\beta} \subseteq A_{\beta}$. So $U \cap A_{\gamma} \neq \emptyset$ and $A_{\gamma} \subseteq U$ as well. Thus $A \subseteq U$ and $V \cap A = \emptyset$. We conclude that $A = \bigcup_{\alpha \in I} A_{\alpha}$ is connected.  
\end{proof}
 
A consequence of \ref{lem:Connected_union} is the following.

\begin{corollary} \label{cor:Connected_union} Let $X$ be a topological space, and let $A_{\alpha}$ be a connected subset of $X$ for all $\alpha$ in some indexing set $I$. Let $B$ be a connected subset of $X$ such that $A_{\alpha} \cap B \neq \emptyset$ for every $\alpha \in I$. Then $B \cup \left(\bigcup_{\alpha \in I} A_{\alpha} \right)$ is connected. 
\end{corollary}

\begin{proof} Let $X$ be a topological space, and let $A_{\alpha}$ be a connected subset of $X$ for all $\alpha$ in some indexing set $I$. Let $B$ be a connected subset of $X$ such that $A_{\alpha} \cap B \neq \emptyset$ for every $\alpha \in I$. For each $\alpha \in I$ let $B_{\alpha} = B \cup A_{\alpha}$. Let $\beta \in I$. Since $B \cap A_{\beta} \neq \emptyset$, Lemma \ref{lem:Connected_union} shows that $B_{\beta}$ is connected. Given that $B$ is not empty, and $B \subseteq \bigcap_{\alpha \in I} B_{\alpha}$, we see that $\bigcap_{\alpha \in I} B_{\alpha} \neq \emptyset$. Lemma \ref{lem:Connected_union} allows us to conclude that $\bigcup_{\alpha \in I} B_{\alpha}$ is connected. But 
\[\bigcup_{\alpha \in I} B_{\alpha} = \bigcup_{\alpha \in I} (B \cup A_{\alpha}) = B \cup \left(\bigcup_{\alpha \in I} A_{\alpha}\right),\]
and so $B \cup \left(\bigcup_{\alpha \in I} A_{\alpha}\right)$ is connected. 
\end{proof}

We will use Corollary \ref{cor:Connected_union} to show that a product of connected spaces is connected. 
 
\begin{theorem} If $(X, \tau_X)$ and $(Y, \tau_Y)$ are connected topological spaces, then $X \times Y$ with the product topology is a connected topological space.  
\end{theorem}

\begin{proof} Assume $(X, \tau_X)$ and $(Y, \tau_Y)$ are connected topological spaces. Our approach to proving that $X \times Y$ is connected is to write $X \times Y$ as a union of two connected subspaces whose intersection is not empty. Let $a \in X$. The space $X_a = \{a\} \times Y$ is homeomorphic to $Y$ via the inclusion map $i$ which sends $(a,t) \in \{a\} \times Y$ to the point $t \in Y$. Since $Y$ is connected, so is $X'$. Let $b \in Y$. The space $Y_b = X \times \{b\}$ is homeomorphic to $X$ via the inclusion map $i$ which sends $(s,b) \in X \times \{b\}$ to the point $s \in X$. Since $X$ is connected, so is $Y_b$. (The verification of these homeomorphisms is left to the reader.) The point $(a,b)$ is in $X_a \cap Y_b$, so $X_a \cap Y_b \neq \emptyset$ for every $b \in Y$. It follows that $X_a \cup \left( \bigcup_{t \in Y} Y_t \right)$ is connected by Corollary \ref{cor:Connected_union}. All that remains is to prove that $X_a \cup \left( \bigcup_{t \in Y} Y_t \right) = X \times Y$ and we will have demonstrated that $X \times Y$ is connected.  The fact that $X_a \subseteq X \times Y$ and $Y_t \subseteq X \times Y$ for every $t \in Y$ implies that $X_a \cup \left( \bigcup_{t \in Y} Y_t \right) \subseteq X \times Y$. It then remains to show that $X \times Y \subseteq X_a \cup \left( \bigcup_{t \in Y} Y_t \right)$. Let $(u,v) \in X \times Y$. Then $u \in X$ and $v \in Y$ and $(u,v) \in Y_v$. Thus, $X \times Y \subseteq X_a \cup \left( \bigcup_{t \in Y} Y_t \right)$ and so $X \times Y = X_a \cup \left( \bigcup_{t \in Y} Y_t \right)$. Therefore, $X \times Y$ is connected.  
\end{proof}

To complete this section we prove that a product of compact spaces is compact.  

\begin{theorem} If $(X, \tau_X)$ and $(Y, \tau_Y)$ are compact topological spaces, then $X \times Y$ with the product topology is a compact topological space.  
\end{theorem}

\begin{proof} Let $(X, \tau_X)$ and $(Y, \tau_Y)$ be compact topological spaces. Let $\C = \{O_{\alpha}\}$ be an open cover of $X \times Y$ for $\alpha$ in some indexing set $I$. Let $a \in X$ and let $Y_a = \{a\} \times Y$. Since $Y_a$ is homeomorphic to $Y$, we know that $Y_a$ is compact. The collection $\{O_{\alpha} \cap Y_a\}$ is an open cover of $Y_a$, and so has a finite sub-cover $\{O_{\alpha_i}\}_{1 \leq i \leq n}$. The set $N_a = \bigcup_{1 \leq i \leq n} O_{\alpha_i}$ is an open set that contains $Y_a$. We will show that there is a neighborhood $W_a$ of $a$ that $N_a$ contains the entire set $W_a \times Y$. 

Cover the set $Y_a$ with open sets that are contained in $N_a$ (since $N_a$ is open, we can intersect any open set with $N_a$ and still have an open set). Each open set is a union of basis elements, so we can cover $Y_a$ with basis elements $U \times V$ that are contained in $N_a$. Since $Y_a$ is compact, there is a finite collection $U_1 \times V_1$, $U_2 \times V_2$, $\ldots$, $U_m \times V_m$ of basis elements contained in $N_a$ that cover $Y_a$. Assume that each $U_i \times V_i$ intersects $Y_a$ (otherwise, we can remove that set and still have a cover). Let $W_a = U_1 \cap U_2 \cap \cdots \cap U_m$.  Since $a \in U_i$ for each $i$, we know that $W_a$ is not empty. Each $U_i$ is open in $X$ and so $W_a$ is open in $X$. Thus, $W_a$ is a neighborhood of $a$ in $X$. Now we demonstrate that $W_a \times Y \subseteq \bigcup_{1 \leq i \leq m} U_i \times V_i$. Let $(x,y) \in W_a \times Y$. Since the collection $\{U_i \times V_i\}_{1 \leq i \leq m}$ covers $Y_a$, the point $(a,y)$ is in $U_k \times V_k$ for some $k$ between $1$ and $m$. So $y \in Y_k$.  But $x \in W_a = \bigcap_{1 \leq i \leq m} U_i$, so $x \in U_k$. Thus, $(x,y) \in U_k \times V_k$ and we conclude that $W_a \times Y \subseteq \bigcup_{1 \leq i \leq m} U_i \times V_i$.

So for each $a \in X$, the set $N_a$ contains a set of the form $W_a \times Y$, where $W_a$ is a neighborhood of $a$ in $X$. So $W_a \times Y$ is covered by a finite sub-cover of our open cover $\C$ of $X \times Y$. The collection $\{W_a \times Y\}_{a \in X}$ is an open cover of $X \times Y$. Since $X$ is compact, the is a finite sub-cover $W_1$, $W_2$, $\ldots$, $W_r$ of the open cover $\{W_a\}_{a \in X}$ of $X$. is an open cover of $X$. It follows that the sets $W_1 \times Y$, $W_2 \times Y$, $\ldots$, $W_r \times Y$ is a cover of $X \times Y$. For each $i$, the set $W_i \times Y$ is covered by finitely many of the sets in $\C$, and so the collection of these sets forms a finite sub-cover of $X \times Y$ in $\C$. Therefore, $X \times Y$ is compact. 
\end{proof}


