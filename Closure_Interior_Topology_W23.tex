\achapter{1}{Closure and Interior in Topological Spaces}\label{chap:Closure_interior_topology}


\vspace*{-17 pt}
\framebox{\hspace*{3 pt}
\parbox{4.7 in}{\begin{fqs}
\item What does it mean for a set to be closed in a topological space?
\item What important properties do closed sets have in relation to unions and intersections?
\item What is a sequence in a topological space?
\item What does it mean for a sequence to converge in a topological space?
\item What is a limit of a sequence in a topological space?
\item What is a limit point of a subset of a topological space? How are closed sets related to limit points? 
\item What is a boundary point of a subset of a topological space and what is the boundary of a subset of a topological space? How are closed sets related to boundary points? 
\item What does it mean for a space to be Hausdorff? What important properties do Hausdorff spaces have?
\end{fqs}} \hspace*{3 pt}}

\vspace*{13 pt}

\csection{Introduction}\label{sec_closure_int_top_intro}
We defined a closed set in a metric space to be the complement of an open set. Since a topology is defined in terms of open sets, we can make the same definition of closed set in a topological space. With the definition of closed set in hand, we can then ask if it is possible to define limit points, boundary, and closure in topological spaces and determine if there are corresponding properties for these ideas in topological spaces. 

We defined a closed set in a metric space to be the complement of an open set. Since a topology is defined in terms of open sets, we can make the same definition of closed set in a topological space. With the definition of closed set in hand, we can then ask if it is possible to define limit points, boundary, and closure in topological spaces and determine if there are corresponding properties for these ideas in topological spaces. 

\section*{Closed Sets in Topological Spaces}

We define closed sets in topological spaces just as we did in metric spaces.  



\begin{definition} A subset $C$ of a topological space $X$ is \textbf{closed} if its complement $X \setminus C$ is open. 
\end{definition}



\begin{pa}  ~
\be
\item List all of the closed sets in the indicated topological space.
	\ba
	\item $(X, \tau)$ with $X= \{a,b,c,d\}$ and $\tau = \{\emptyset, \{a\}, \{b\}, \{a,b\}, X \}$.



\begin{comment}

\solution The closed sets are the complements of the open sets, so the closed sets are 
\[X, \{b,c,d\}, \{a,c,d\}, \{c,d\}, \text{ and } \emptyset.\]



\end{comment}


	\item $(X, \tau)$ with $X= \{a,b,c,d,e,f\}$ and $\tau = \{\emptyset,\{a\}, \{c,d\}, \{a,c,d\}, \{b,c,d,e,f\}, X\}$.



\begin{comment}

\solution The closed sets are the complements of the open sets, so the closed sets are 
\[X, \{b,c,d,e,f\}, \{a,b,e,f\}, \{b,e,f\}, \{a\}, \text{ and } \emptyset.\]



\end{comment}

	\item $(X, \tau)$ with $X = \R$ and $\tau = \{\emptyset, \{0\}, \R\}$. 



\begin{comment}

\solution The closed sets are the complements of the open sets, so the closed sets are 
\[\R, \R-\{0\}, \text{ and } \emptyset.\]



\end{comment}

	\item $(X, \tau)$ with $X = \{a,b,c\}$ and $\tau = \{\emptyset, \{a\}, \{b\},\{c\}, \{a,b\}, \{a,c\}, \{b,c\}, X \}$. (What is the name of this topology?)



\begin{comment}

\solution  Every subset is open, so this topology is the discrete topology. The closed sets are the complements of the open sets, so every subset is also closed. So the closed sets are 
\[X, \{b,c\}, \{a,c\}, \{a,b\}, \{c\}, \{b\},\{a\}, \text{ and } \emptyset.\]



\end{comment}

	\item $(X, \tau)$ with $X=\Z^+$ and $\tau = \{\emptyset, X\}$ (this topology is called the \emph{indiscrete} or \emph{trivial} topology). 



\begin{comment}

\solution The closed sets are the complements of the open sets, so the closed sets are 
\[X  \text{ and } \emptyset.\]



\end{comment}

	\ea


\item In each of the examples from part (1), find (if possible), a set that is
	\ba
	\item both closed and open (if possible, find one that is not the entire set or the empty set)
	
	

\begin{comment}

\solution In the topological space $(X, \tau)$ with $X= \{a,b,c,d,e,f\}$ and $\tau = \{\emptyset,\{a\}, \{c,d\}, \{a,c,d\}, \{b,c,d,e,f\}, X\}$ the set $\{a\}$ is both open and closed. 



\end{comment}
	
	\item closed but not open
	
	

\begin{comment}

\solution In the topological space $(X, \tau)$ with $X= \{a,b,c,d\}$ and $\tau = \{\emptyset, \{a\}, \{b\}, \{a,b\}, X \}$ the set $\{a,c,d\}$ is closed but not open. 



\end{comment}
	
	\item open but not closed
	
	

\begin{comment}

\solution In the topological space $(X, \tau)$ with $X= \{a,b,c,d\}$ and $\tau = \{\emptyset, \{a\}, \{b\}, \{a,b\}, X \}$ the set $\{a,b\}$ is open but not closed. 



\end{comment}
	
	\item not open and not closed
	
	

\begin{comment}

\solution Consider the topological space $(X, \tau)$ with $X= \{a,b,c,d\}$ and $\tau = \{\emptyset, \{a\}, \{b\}, \{a,b\}, X \}$. The set $\{b,c\}$ is neither open nor closed. 



\end{comment}
	
	\ea

\ee

One characterization of closed sets in metric spaces was that a closed set contained its limit points. Recall that limit points were defined in terms of neighborhoods, so we can make the same definition in topological spaces. 



\begin{definition} Let $X$ be a topological space, and let $A$ be a subset of $X$. A \textbf{limit point} of $A$ is a point $x \in X$ such that every neighborhood of $x$ contains a point in $A$ different from $x$. 
\end{definition}



The set $A'$ of limit points of $A$ is called the \emph{derived set} of $A$. 



\be 
\item[3.] Find the limit point(s) of the following sets
	\ba
	\item $\{c,d\}$ in $(X, \tau)$ with $X= \{a,b,c,d\}$ and $\tau = \{\emptyset, \{a\}, \{b\}, \{a,b\}, X \}$.



\begin{comment}

\solution Neither $a$ nor $b$ is a limit point of $\{c,d\}$ since the open neighborhood $\{a,b\}$ contains no point in $\{c,d\}$ different than $a$ or $b$. The only open set that contains $c$ or $d$ is $X$, so that is the only neighborhood of $c$ or $d$. Since $X$ contains a point in $\{c,d]{$ that is different than $c$ (or $d$), both $c$ and $d$ are limit points of $\{c,d\}$. 




\end{comment}


	\item $\{a,b\}$ in the set $X= \{a,b,c,d,e,f\}$ with topology $\tau= \{\emptyset,\{b\}, \{a,b,c\},\{d,e,f\},\{b,d,e,f\}, X\}$ 



\begin{comment}

\solution None of the points $b$, $d$, $e$, or $f$ is a limit point of $\{a,b\}$ since the open neighborhood $\{b,d,e,f\}$ contains no point in $\{a,b\}$ different than $b$, $d$, $e$, or $f$. Any neighborhood of $a$ or $c$ must contain one of the open sets $\{a,b,c\}$ or $X$. So every neighborhood of $a$ or $c$ contains a point of $\{a,b\}$ different than $a$ or $c$. Therefore, the limit points of $\{a,b\}$ are $a$ and $c$ and $\{a,b\}' = \{a,c\}$. 




\end{comment}


	\item $\{a,b\} \subset X$ where $X = \{a,b,c\}$ in the discrete topology. 



\begin{comment}

\solution For any $x \in \{a,b\}$, the open neighborhood $\{x\}$ of $x$ does not contain any points in $\{a,b\}$ different than $x$. So the set $\{a,b\}$ has no limit points. 



\end{comment}
 

\ea

\ee



\end{pa}

\csection{Closed Sets in Topological Spaces}\label{sec_closed_sets_top_spaces}

We defined closed sets in topological spaces in our Preview Activity. 



\begin{definition} A subset $C$ of a topological space $X$ is \textbf{closed} if its complement $X \setminus C$ is open. 
\end{definition}



We saw in our Preview Activity that a set can be both open and closed in a topological space. We call such sets \emph{clopen} (for closed-open). 



In metric spaces, arbitrary intersections and finite unions of closed sets were closed. The fact that closed sets in topological spaces are complements of open sets implies the same result in topological spaces. The proof is left to the exercises.  



\begin{theorem} Let $X$ be a topological space.
\begin{enumerate}
\item Any intersection of closed sets in $X$ is a closed set in $X$.
\item Any finite union of closed sets in $X$ is a closed set in $X$. 
\end{enumerate}
\end{theorem}

%


%\begin{proof} Let $X$ be a topological space. To prove part 1, assume that $\{C_{\alpha}\}$ is a collection of closed set in $X$ for $\alpha$ in some indexing set $I$. Then 
%\[X \setminus \bigcap_{\alpha \in I} C_{\alpha} = \bigcup_{\alpha \in I} X \setminus C_{\alpha}.\]
%The latter is an arbitrary union of open sets and so is an open set. By definition, then, $\bigcap_{\alpha \in I} C_{\alpha}$ is a closed set. 

%The proof of part 2 is left for the following activity.
%\end{proof}

%

%\begin{activity} Let $X$ be a topological space, and let $C_1$, $C_2$, $\ldots$, $C_n$ be closed sets in $X$ for some $n \in \Z^+$. Prove that 
%\[C = \bigcap_{k=1}^n C_k\]
%is a closed set in $X$. 

%

%\begin{comment}

%For part 2, assume that $C_1$, $C_2$, $\ldots$, $C_n$ are closed sets in $X$ for some $n \in \Z^+$. To show that $C = \bigcap_{k=1}^n C_k$ is a closed set, we will show that $X \setminus C$ is an open set. Now 
%\[X \setminus \bigcup_{\alpha \in I} C_{\alpha} = \bigcap_{\alpha \in I} X \setminus C_{\alpha}\]
%is a finite intersection of open sets, and so is an open set. Therefore, $\bigcup_{\alpha \in I} C_{\alpha} $ is a closed set. 
%\end{comment}

%\end{activity}


\csection{Limit Points and Closure in Topological Spaces}\label{sec_limit_closure}

As discussed in our Preview Activity, limit points in metric spaces were defined in terms of neighborhoods, so we can make the same definition in topological spaces. 



\begin{definition} Let $X$ be a topological space, and let $A$ be a subset of $X$. A \textbf{limit point} of $A$ is a point $x \in X$ such that every neighborhood of $x$ contains a point in $A$ different from $x$. 
\end{definition}



In metric spaces, a set is closed if and only if it contains all of its limit points. So the corresponding result in topological spaces should be no surprise. 



\begin{theorem} \label{thm:TS_closed_limitpoints} Let $C$ be a subset of a topological space $X$, and let $C'$ be the set of limit points of $C$. Then $C$ is closed if and only if $C' \subseteq C$.  
\end{theorem}

\begin{proof} Let $X$ be a topological space, and let $C$ be a subset of $X$. First we assume that $C$ is closed and show that $C$ contains all of its limit points. Let $x \in X$ be a limit point of $C$. We proceed by contradiction and assume that $x \notin C$. Then $x \in X \setminus C$, which is an open set. This means that there is a neighborhood (namely $X \setminus C$) of $x$ that contains no points in $C$, which contradicts the fact that $x$ is a limit point of $C$. We conclude that $x \in C$ and $C$ contains all of its limit points.

The proof of the converse is left for you in the next activity. 
%For the converse, assume that $C$ contains all of its limit points. To show that $C$ is closed, we prove that $X \setminus C$ is open. We again proceed by contradiction and assume that $X \setminus C$ is not open. Then there exists $x \in X \setminus C$ such that no neighborhood of $x$ is entirely contained in $X \setminus C$. This implies that every neighborhood of $x$ contains a point in $C$ and so $x$ is a limit point of $C$. It follows that $x \in C$, contradicting the fact that $x \in X \setminus C$. We conclude that $X \setminus C$ is open and $C$ is closed.
\end{proof}



\begin{activity} Let $C$ be a subset of a topological space $X$, and let $C'$ be the set of limit points of $C$. Assume $C' \subseteq C$.
	\ba
	\item How can we prove that $C$ is closed in $X$? 
	
	
	
	\item Proceed by contradiction and show that $C$ is closed. 
	
	
	
	\ea
	
\end{activity}



Once we have a definition of limit point, we can define the closure of a set just as we did in metric spaces. 



\begin{definition} The \textbf{closure} of a subset $A$ of a topological space $X$ is the set 
\[\overline{A} = A \cup A'.\]
\end{definition}



In other words, the closure of a set is the collection of the elements of the set and the limit points of the set. The following theorem is the analog of the theorem in metric spaces about closures. 



\begin{theorem} \label{thm:TS_closure_closed} Let $X$ be a topological space and $A$ a subset of $X$. The closure of a $A$ is a closed set. Moreover, the closure of $A$ is the smallest closed subset of $X$ that contains $A$. 
\end{theorem}

\begin{proof} Let $X$ be a topological space and $A$ a subset of $X$. To prove that $\overline{A}$ is a closed set, we will prove that $\overline{A}$ contains its limit points. Let $x \in \overline{A}'$. To show that $x \in \overline{A}$, we proceed by contradiction and assume that $x \notin \overline{A}$. This implies that $x \notin A$ and $x \notin A'$. Since $x \notin A'$, there exists a neighborhood $N$ of $x$ that contains no points of $A$ other than $x$. But $A \subseteq \overline{A}$ and $x \notin \overline{A}$, so it follows that $N \cap A = \emptyset$. This implies that there is an open set $O \subseteq N$ centered at $x$ so that $O \cap A = \emptyset$. The fact that $x \in \overline{A}'$ means that $O \cap \overline{A}$ contains a point $y$ in $\overline{A}$ different from $x$. Since $O \cap A = \emptyset$, we must have $y \in A'$. But the fact that $O$ is a neighborhood of $y$ means that $O$ must contain a point of $A$ different than $y$, which contradicts the fact that $O \cap A = \emptyset$. We conclude that $x \in \overline{A}$ and $\overline{A}' \subseteq \overline{A}$. This shows that $\overline{A}$ is a closed set. 

The proof that $\overline{A}$ is the smallest closed subset of $X$ that contains $A$ is left for the next activity.
\end{proof}



\begin{activity} Let $(X,d)$ be a topological space, and let $A$ be a subset of $X$. 
\ba
\item What will we have to show to prove that $\overline{A}$ is the smallest closed subset of $X$ that contains $A$?



\item Suppose that $C$ is a closed subset of $X$ that contains $A$. To show that $\overline{A} \subseteq C$, why is it enough to demonstrate that $A' \subseteq C$? 



\item If $x \in A'$, what can we say about $x$? 



\item Complete the proof that $\overline{A} \subseteq C$.



\begin{comment}

We need to prove that any closed subset of $X$ that contains $A$ also contains $\overline{A}$. Suppose that $C$ is a closed subset of $X$ that contains $A$. To show that $C$ contains $\overline{A}$, we need only to show that $C$ contains $A'$. Let $x \in A'$, and let $N$ be a neighborhood of $x$. Then $N$ contains a point of $A$ different than $x$. Since $A \subseteq C$, it follows that $N$ contains a point of $C$ different than $x$. So $x$ is a limit point of $C$. The fact that $C$ is closed means that $C$ contains its limit points, so $x \in C$. Therefore, $A' \subseteq C$ and $\overline{A} \subseteq C$. 
\end{comment}

\ea

\end{activity}



Two consequences of Theorem \ref{thm:TS_closure_closed} are the following.



\begin{corollary} A subset $C$ of a topological space $X$ is closed if and only if $C = \overline{C}$. 
\end{corollary}



\begin{corollary} Let $X$ be a topological space and $A$ a subset of $X$. Then 
\[\overline{A} = \bigcap_{\alpha \in I} C_{\alpha},\]
where $\{C_{\alpha}\}_{\alpha \in I}$ is the collection of closed sets that contain $A$. 
\end{corollary}

\begin{proof} Let $X$ be a topological space, and $A$ a subset of $X$. Let $\{C_{\alpha}\}_{\alpha \in I}$ be the collection of closed sets that contain $A$. We know that $\overline{A}$ is a subset of each $C_{\alpha}$, so $\overline{A} = \bigcap_{\alpha \in I} C_{\alpha}$. Also, $\overline{A}$ is closed and contains $A$, so $\overline{A}$ is in $\{C_{\alpha}\}_{\alpha \in I}$. This makes $\bigcap_{\alpha \in I} C_{\alpha} \subseteq \overline{A}$. Therefore, $\overline{A} = \bigcap_{\alpha \in I} C_{\alpha}$. 
\end{proof} 



\csection{The Boundary of a Set}\label{sec_boundary_set}

In addition to limit points, we also defined boundary points in metric spaces. Recall that a boundary point of a set $A$ in a metric space $X$ could be considered to be any point in $\overline{A} \cap \overline{X \setminus A}$. We make the same definition in a topological space. 



\begin{definition} Let $X$ be a metric space, and let $A$ be a subset of $X$. A \textbf{boundary point} of $A$ is a point $x \in X$ such that every neighborhood of $x$ contains a point in $A$ and a point in $X \setminus A$. The \textbf{boundary} of $A$ is the set 
\[\Bdry(A) = \{x \in X \mid x \text{ is a boundary point of } A\}.\]
\end{definition}



So the boundary points of a set $A$ are those points that are ``between" a set and its complement. 



\begin{activity} \label{act:TS_boundaries} Find the boundaries of the following sets
\ba
\item $\{c,d\}$ in $(X, \tau)$ with $X= \{a,b,c,d\}$ and $\tau = \{\emptyset, \{a\}, \{b\}, \{a,b\}, X \}$.



\begin{comment}

\solution Neither $a$ nor $b$ is a boundary point of $\{c,d\}$ since the open neighborhood $\{a,b\}$ contains no point in $\{c,d\}$. The only open set that contains $c$ or $d$ is $X$, so that is the only neighborhood of $c$ or $d$. Since $X$ contains a point in $\{c,d]{$ that is different than $c$ (or $d$), and a point not in $\{c,d\}$, both $c$ and $d$ are boundary points of $\{c,d\}$. Therefore, $\Bdry(\{c,d\}) = \{c,d\}$. 




\end{comment}



\item $\{a,b\}$ in the set $X= \{a,b,c,d,e,f\}$ with topology $\tau= \{\emptyset,\{b\}, \{a,b,c\},\{d,e,f\},\{b,d,e,f\}, X\}$ 



\begin{comment}

\solution None of the points $d$, $e$, or $f$ is a boundary point of $\{a,b\}$ since the open neighborhood $\{d,e,f\}$ contains no point in $\{a,b\}$. The open neighborhood $\{b\}$ of $b$ contains no points that are not in $\{a,b\}$, so $b$ is not a boundary point of $\{a,b\}$. Any neighborhood of $a$ or $c$ must contain one of the open sets $\{a,b,c\}$ or $X$. So every neighborhood of $a$ or $c$ contains a point of $\{a,b\}$ and a point not in $\{a,b\}$. So $a$ and $c$ are boundary points of $\{a,b\}$. Therefore, $\Bdry(\{a,b\}) = \{a,c\}$.  




\end{comment}


\item $\{a,b\} \subset X$ where $X = \{a,b,c\}$ in the discrete topology. 



\begin{comment}

\solution For any $x \in \{a,b\}$, the open neighborhood $\{x\}$ of $x$ does not contain any points in $\{a,b\}$ different than $x$. So the set $\{a,b\}$ has no limit points. 



\end{comment}
 

\item $(0,1)$ in $(\R, \tau_{FC})$ 



\begin{comment}

\solution Let $x \in \R$. If $N$ is a neighborhood of $x$, then $N$ must contain some open set $O$ that contains $x$. Since $O$ is an open set, it follows that $O \setminus (0,1)$ is a finite set. Also, $O \setminus (\R \setminus (0,1))$ must be finite. So $O$ contains infinitely many points in $(0,1)$ and infinitely many points in $\R \setminus (0,1)$. Therefore, every point in $\R$ is both a limit point and a boundary point of $(0,1)$. 


\end{comment}

\ea

\end{activity}



Just as in metric spaces, closed sets are sets are exactly the sets that contain their boundaries. 



\begin{theorem} \label{thm:TS_Closed_boundary} A subset $C$ of a topological space $X$ is closed if and only if $C$ contains its boundary. 
\end{theorem}

\begin{comment}

\begin{proof} Let $X$ be a topological space, and let $C$ be a subset of $X$. First we assume that $C$ is closed and show that $C$ contains its boundary. Let $x \in X$ be a boundary point of $C$. We proceed by contradiction and assume that $x \notin C$. Then $x \in X \setminus C$, which is an open set. But then this neighborhood $X \setminus C$ contains no points in $C$, which contradicts the fact that $x$ is a boundary point of $C$. We conclude that $x \in C$ and $C$ contains its boundary.

For the converse, assume that $C$ contains its boundary. To show that $C$ is closed, we prove that $C$ contains its limit points. Let $x$ be a limits point of $C$. To show that $x \in C$, assume to the contrary that $x \notin C$. Then $x \in X \setminus C$, an open set. Since $X \setminus C$ is a neighborhood of each of its points, the fact that $x$ is a limit point of $C$ implies that $X \setminus C$ must contain a point of $C$, a contradiction. We conclude that $x \in C$ and $C$ contains its limit points. Therefore, $C$ is closed. 
\end{proof}
\end{comment}



The proof of Theorem \ref{thm:TS_Closed_boundary} is left to the reader.


\csection{The Interior of a Set}\label{sec_interior_set}

Recall that the interior of a subset of a metric space was the collection of interior points -- a point is an interior point of a set if the set is a neighborhood of the point. We have the same concepts in topological spaces, so we can make the same definitions. 



\begin{definition} Let $A$ be a subset of a topological space $X$. A point $a \in A$ is an \textbf{interior point} of $A$ if $A$ is a neighborhood of $a$.
\end{definition}



Open sets can be characterized in terms of their interior points. By definition, every open set is a neighborhood of each of its points, so every point of an open set is an interior point. This argument is summarized in the next theorem. 



\begin{theorem} Let $X$ be a topological space. A subset $O$ of $X$ is open if and only if every point of $O$ is an interior point of $O$. 
\end{theorem}



The collection of interior points in a set form a subset of that set, called the \emph{interior} of the set.



\begin{definition} The \textbf{interior} of a subset $A$ of a topological space $X$ is the set
\[\Int(A) = \{a \in A \mid a \text{ is an interior point of } A\}.\]
\end{definition}



\begin{activity} \label{act:TS_interiors} Find the interiors of the following sets
\ba
\item $\{c,d\}$ in $(X, \tau)$ with $X= \{a,b,c,d\}$ and $\tau = \{\emptyset, \{a\}, \{b\}, \{a,b\}, X \}$.



\begin{comment}

\solution The only open set that contains $c$ or $d$ is $X$, and $X$ is not a subset of $\{c,d\}$. So $\Int(\{c,d\}) = \emptyset$. 




\end{comment}



\item $\{a,b\}$ in the set $X= \{a,b,c,d,e,f\}$ with topology $\tau= \{\emptyset,\{b\}, \{a,b,c\},\{d,e,f\},\{b,d,e,f\}, X\}$ 



\begin{comment}

\solution None of the open sets containing $a$ are subsets of $\{a,b\}$, so $a \notin \Int(\{a,b\})$. The open set $\{b\}$ is a subset of $\{a,b\}$, so $\int(\{a,b\}) = \{b\}$. 




\end{comment}


\item $\{a,b\} \subset X$ where $X = \{a,b,c\}$ in the discrete topology. 



\begin{comment}

\solution The open sets $\{a\}$ and $\{b\}$ are both subsets of $\{a,b\}$, so $\Int(\{a,b\}) = \{a,b\}$. Note that $\{a,b\}$ is an open set. 



\end{comment}
 

\item $(0,1)$ in $(\R, \tau_{FC})$ 



\begin{comment}

\solution Let $x \in (0,1)$. Suppose $O$ is an open set that contains $x$. Since $O$ is an open set, it follows that $O \setminus (\R \setminus (0,1))$ must be finite. So $O$ contains infinitely many points in $\R \setminus (0,1)$. Therefore, the set $(0,1)$ contains no interior points and $\Int((0,1)) = \emptyset$. every point in $\R$ is both a limit point and a boundary point of $(0,1)$. 

\end{comment}

\ea

\end{activity}




One might expect that the interior of a set is an open set. This is true, but we can say even more. 



\begin{theorem} \label{thm:TS_Interior} Let $(X,d)$ be a topological space, and let $A$ be a subset of $X$. The interior of $A$ is the largest open subset of $X$ contained in $A$.  
\end{theorem}

\begin{proof} Let $(X,d)$ be a topological space, and let $A$ be a subset of $X$. We need to prove that $\Int(A)$ is an open set in $X$, and that $\Int(A)$ is the largest open subset of $X$ contained in $A$. First we demonstrate that $\Int(A)$ is an open set. Let $a \in \Int(A)$. Then $a$ is an interior point of $A$, so $A$ is a neighborhood of $a$. This implies that there exists an open set $O$ contained in $A$ such that $a \in O$. But $O$ is a neighborhood of each of its points, so every point in $O$ is an interior point of $A$. It follows that $O \subseteq \Int(A)$. Thus, $\Int(A)$ is a neighborhood of each of its points and, consequently, $\Int(A)$ is an open set. 

The proof that $\Int(A)$ is the largest open subset of $X$ contained in $A$ is left for the next activity.
\end{proof}



\begin{activity} Let $(X,d)$ be a topological space, and let $A$ be a subset of $X$. 
\ba
\item What will we have to show to prove that $\Int(A)$ is the largest open subset of $X$ contained in $A$?



\item Suppose that $O$ is an open subset of $X$ that is contained in $A$, and let $x \in O$. What does the fact that $O$ is open tell us?



\item Complete the proof that $O \subseteq \Int(A)$.



\begin{comment}

We need to prove that any open subset of $X$ that is contained in $A$ is a subset of $\Int(A)$. Suppose that $O$ is an open subset of $X$ that is contained in $A$. Let $x \in O$. Then $O$ is a neighborhood of $x$ and so $x \in \Int(A)$. Therefore, $O \subseteq \Int(A)$ and $\Int(A)$ is the largest open subset of $X$ contained in $A$. 
\end{comment}

\ea

\end{activity}



One consequence of Theorem \ref{thm:TS_Interior} is the following.



\begin{corollary} A subset $O$ of a metric space $X$ is open if and only if $O = \Int(O)$. 
\end{corollary}
 


Since $\Int(A)$ is an open set that contains all of the open subsets of $A$, we also have the following.



\begin{corollary} \label{cor:int_union} Let $X$ be a topological space and $A$ a subset of $X$. Then 
\[\Int(A) = \bigcup_{\alpha \in I} O_{\alpha},\]
where $\{O_{\alpha}\}_{\alpha \in I}$ is the collection of open subsets of $A$. 
\end{corollary}



A consequence of Corollary \ref{cor:int_union} is the following.



\begin{corollary} Let $X$ be a topological space and $A$ a subset of $X$. Then
\[X \setminus \Int(A) = \overline{X \setminus A}.\]
\end{corollary}

\begin{proof} Let $\{O_{\alpha}\}_{\alpha \in I}$ be the collection of open subsets of $A$. Then $\{X \setminus O_{\alpha}\}_{\alpha \in I}$ is the collection of closed subsets of $X$ that contain $X \setminus A$. So
\[X \setminus \Int(A) = X \setminus \bigcup_{\alpha \in I} O_{\alpha} = \bigcap_{\alpha \in I} X \setminus O_{\alpha} = \overline{X \setminus A}.\]
\end{proof}   




\csection{Sequences in Topological Spaces}\label{sec_seq_top_spaces}

One property of closed sets in metric spaces was that closed sets contained limits of convergent sequences. We defined a sequence in a metric space as a function from the positive integers to the space, and we can apply the same definition in topological spaces. In addition, we defined convergence of sequences in a metric space, and were able to phrase this concept in terms of open sets. Consequently, we can make a similar definition in topological spaces. As we will see, though, the notion of convergent sequence is less useful in general topological spaces than in metric spaces. 



\begin{definition} A \textbf{sequence} in a topological space $X$ is a function $f: \Z^+$ to $X$.
\end{definition}



We use the same notation and terminology related to sequences as we did in metric spaces: we will write $(x_n)$ to represent a sequence $f$, where $x_n = f(n)$ for each $ n \in \Z^+$. We can't define convergence in a topological space using a metric, but we can also use open sets. 



\begin{definition} A sequence $(x)_n$ in a topological space $X$ \textbf{converges} to the point $x \in X$ if, for each open set $O$ that contains $x$ there exists a positive integer $N$ such that $x_n \in O$ for all $n \geq N$. 
\end{definition}



If a sequence $(x_n)$ converges to a point $x$, we call $x$ a \emph{limit} of the sequence $(x_n)$. Recall that limits of convergent sequences are unique in a metric space. It is reasonable to expect that this result is also true in topological spaces. 



\begin{activity} \label{act:TS_limits} Consider the topological space $(\R, \tau_{FC})$. (Recall that $\tau_{FC}$ is the finite complement topology consisting of the empty set and the subsets $O$ of $\R$ such that $\R \setminus O$ is finite.) Let $(x)_n$ be a sequence of distinct points in $\R$. 
	\ba
	\item Explain why every open set in $\R$ must contain infinitely many elements of the sequence $(x_n)$. Why does it follow that if $O$ is an open set in $\R$, there exists $N \in \Z^+$ so that $x_n \in O$ for all $n \geq N$?



	\item What does the result of part (a) tell us about the sequence $(x_n)$? 



	\ea
	
%\item Let $X = \{a,b\}$ with the trivial topology $\tau = \{\emptyset, X\}$. 
%	\begin{enumerate}[i.]
%	\item Is $b$ a limit of the sequence $(b)$? Why or why not? 
	
%	
	
%	\item Does the sequence $(b)$ converge in $X$? If no, why not? If yes, what is a limit of $(b)$? 
	
%	
	
%	\end{enumerate}

\end{activity}


The result of Activity \ref{act:TS_limits} is that sequences do not play the same important role in general topological spaces as they do in metric spaces. However, the concept of limit point is important, as it defines the closure of a set. 
