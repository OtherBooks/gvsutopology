\achapter{11}{Subspaces and Products of Metric Spaces}\label{sec:metric_subspaces}


\vspace*{-17 pt}
\framebox{
\parbox{\dimexpr\linewidth-3\fboxsep-3\fboxrule}
{\begin{fqs}
\item What is a subspace of a metric space? 
\item How do we find the open and closed sets in a subspace of a metric space?
\item What is a product of metric spaces and how do we make a product of metric spaces into a metric space?
\end{fqs}}}

\vspace*{13 pt}

\csection{Introduction}

Let $(X,d)$ be a metric space, and let $A$ be a subset of $X$. We can make $A$ into a metric space itself in a very straightforward manner. When we do so, we say that $A$ is a \emph{subspace} of $X$. 


\begin{pa}  Let $(X,d)$ be a metric space, and let $A$ be a subset of $X$. To make the subset $A$ into a metric space, we need to define a metric on $A$. For us to consider $A$ as a subspace of $X$, we want the metric on $A$ to agree with the metric on $X$. So we define $d' : A \times A \to \R$ by 
\[d'(a_1,a_2) = d(a_1,a_2)\]
for all $a_1, a_2 \in A$. 
Note that $d$ and $d'$ are different functions because they have different domains. 
\begin{enumerate}
\item Show that $d'$ is a metric on $A$.

\vspace{0.25in}

Since $d'$ is a metric on $A$ it follows that $(A,d')$ is a metric space. The metric $d'$ is the \emph{restriction} of $d$ to $A \times A$ and can also be denoted by $d_A$. 

\begin{definition} Let $(X,d)$ be a metric space. A \textbf{subspace}\index{subspace of a metric space} of $(X,d)$ is a subset $A$ of $X$ together with the metric $d_A$ from $A \times A$ to $\R$ defined by 
\[d_A(a_1,a_2) = d(a_1,a_2)\]
for all $a_1,a_2 \in A$.  
\end{definition}

We might wonder what, if any, properties of the space $X$ are inherited by a subspace.

\vspace{0.25in}

\item Let $(X,d) = (\R,d_E)$ and let $A = [0,1]$. Let $0 < a < 1$. Is the set $[0,a)$ open in $X$? Is the set $[0,a)$ open in $A$? Explain.

\item Let $(X,d) = (\R,d_E)$ and let $A = \Z$. What are the open subsets of $A$? Explain. 

\item  Let $(X,d) = (\R^2,d_E)$, let $A = \{(x,0) \mid x \in \R\}$ (the $x$-axis in $\R^2$), and let $Z = \{(x,0) \mid 0 < x < 1\}$. Note that $Z \subset A \subset X$ and we can consider $Z$ as a subspace of $A$ and $X$, and $A$ as a subspace of $X$. 
		\ba
		\item Explain why $A$ is a closed subset of $X$.
	
		\item Explain why $Z$ is an open subset of $A$.

		\item Is $Z$ an open subset of $X$? Explain

		\ea
	
	\end{enumerate}
	
\end{pa}


\begin{comment}

\ActivitySolution

\begin{enumerate}
\item Let $a_1, a_2$, and $a_3$ be in $A$. Now $d'(a_1,a_2) = d(a_1,a_2) \geq 0$ with equality if and only if $a_1=a_2$ since $d$ is a metric. Also, $d'(a_1,a_2) = d(a_1,a_2) =  d(a_2,a_1) = d'(a_2,a_1)$, so $d'$ is symmetric. Finally, the fact that $d$ is a metric shows that 
\[d'(a_1,a_3) = d(a_1,a_3) \leq d(a_1,a_2) + d(a_2,a_3) = d'(a_1,a_2) + d'(a_2,a_3),\]
and $d'$ satisfies the triangle inequality. Thus, $d'$ is a metric on $A$. 

\item Since the set $[0,a)$ is not a neighborhood of $0$ in $X$, the set $[0,a)$ is not open in $X$. However, in $A$, the set $[0,a)$ always contains the open ball $B(r,\min\{r,a-r\})$ for any $r \in (0,a)$, and also contains the open ball $[0,\frac{a}{2}) = B\left(0, \frac{a}{2}\right)$.  So $[0,a)$ is an open neighborhood of each of its points in $A$ and is therefore an open set in $A$. So the open sets in a subspace of $X$ can be different than the open sets in $X$. 

\item  If $a \in \Z$, then $B(a,0.5) = \{a\}$ in $\Z$. So any point is an interior point of a subset of $\Z$ that contains it. Thus, every subset of $\Z$ is open in $\Z$. Note also, that $d_E$ restricted to $\Z$ is the discrete metric. 

\item
		\ba
		\item  Note that $X \setminus A = \{(x,y) \mid y \neq 0\}$. If $(a,b) \in X \setminus A$, then $B\left((a,b), \frac{b}{2}\right) \subset X \setminus A$. Thus, $X \setminus A$ is open in $X$ and so $A$ is closed in $X$. 

		\item  If $z \in Z$, then the open ball $B(z,\min\{z, 1-z\})$ is contained in $Z$. Thus, $Z$ is open in $A$.

		\item The answer is no. Let $z \in Z$ and let $r > 0$. The open ball $B(z,r)$ contains the point $\left(z, \frac{r}{2}\right)$ which is not in $Z$. So $Z$ is a neighborhood of none of its points in $X$ and is therefore not open in $X$. 

		\ea
	
\end{enumerate}

\end{comment}


\csection{Open and Closed Sets in Subspaces}

We saw in our preview activity that a subspace does not necessarily inherit the properties of the larger space. For example, a subset of a subspace might be open in the subspace, but not open in the larger space. However, there is a connection between the open subsets in a subspace and the open sets in the larger space. 

\begin{theorem} \label{thm:relatively_open_ms} Let $(X,d)$ be a metric space and $A$ a subset of $X$. A subset $O_A$ of $A$ is open in $A$ if and only if there is an open set $O_X$ in $X$ so that $O_A = O_X \cap A$. 
\end{theorem}

\begin{proof} Let $(X,d)$ be a metric space and $A$ a subset of $X$. Let $O_A$ be an open subset of $A$. So for each $a \in A$ there is a $\delta_a > 0$ so that $B_A(a, \delta_a) \subseteq O_A$, where $B_A(a, \delta_a)$ is the open ball in $A$ centered at $a$ of radius $\delta_a$. Then, $O_A~=~\bigcup_{a \in O_A}~B_A(a, \delta_a)$. Now let $B_X(a, \delta_a)$ be the open ball in $X$ centered at $a$ of radius $\delta_a$, and let $O_X = \bigcup_{a \in O_A} B_X(a, \delta_a)$. Note that \[B_A(a, \delta_a)~=~B_X(a, \delta_a)~\cap~A.\]
As a union of open balls in $X$, the set $O_X$ is open in $X$. Now
\[O_X \cap A = \left(\bigcup_{a \in O_A} B_X(a, \delta_a) \right) \cap A = \bigcup_{a \in O_A} \left( B_X(a, \delta_a) \cap A \right) = \bigcup_{a \in O_A} B_A(a, \delta_a) = O_A.\]
So there is an open set $O_X$ in $X$ such that $O_A = O_X \cap A$. 

For the reverse implication, see the following activity. 
\end{proof}


\begin{activity} Let $(X,d)$ be a metric space and $A$ a subset of $X$. Suppose that $O_A = A \cap O_X$ for some set $O_X$ that is open in $X$. In this activity we will prove that $O_A$ is an open subset of $A$. 
\ba
\item Let $a \in O_A$. Explain why there must exist a $\delta > 0$ such that $B_X(a, \delta)$, the open ball in $X$ of radius $\delta$ around $a$ in $X$, is a subset of $O_X$. 

\item What would be a natural candidate for an open ball in $A$ centered at $a$ that is contained in $A$? Prove your conjecture.

\item What conclusion can we draw?

\ea

\end{activity} 

\begin{comment}

\ActivitySolution

\ba
\item Since $O_A = O_X \cap A$, we also have $a \in O_X$. The fact that $O_X$ is open means that there exists $\delta > 0$ such that $B_X(a, \delta)$ is a subset of $O_X$. 

\item Let $B_A(a, \delta) = B_X(a, \delta) \cap A$. We will show that $B_A(a, \delta)$ is the open ball $B$ in $A$ centered at $a$ of radius $\delta$. If $t \in B_A(a, \delta)$, then $t \in A$ and $t \in B_X(a, \delta)$. So $t \in A$ and $d|_A(t,a) < \delta$. So $t \in B$. Now suppose $t \in B$. Then $t \in A$ and $d|_A(t,a) = d(t,a) < \delta$. Thus, $t \in A$ and $t \in B_X(a, \delta)$. This means that $t \in B_X(a, \delta) \cap A = _A(a, \delta)$. Thus, $B_A(a, \delta) = B$ and $B_A(a, \delta)$ is the open ball in $A$ centered at $a$ of radius $\delta$.

\item Since $B_A(a, \delta)$ is an open set in $A$, we conclude that $O_A$ is a neighborhood of each of its points and $O_A$ is open in $A$. 

\ea

\end{comment}

We might now wonder about closed sets in a subspace. If $X$ is a metric space and $A$ is a subspace, then by definition a subset $C_A$ of $A$ is closed if and only if $C_A = A \setminus O_A$ for some set $O_A$ that is open in $A$. The  analogy of Theorem \ref{thm:relatively_open_ms} is true for closed sets in subspaces.

\begin{theorem} \label{thm:relatively_closed_ms} Let $(X,d)$ be a metric space and $A$ a subset of $X$. A subset $C_A$ of $A$ is closed in $A$ if and only if there is a closed set $C_X$ in $X$ so that $C_A = C_X \cap A$. 
\end{theorem}

The proof is left to Exercise (\ref{ex:relatively_closed_ms}).


\csection{Products of Metric Spaces}
If we have two metric spaces $(X, d_1)$ and $(X_2 ,d_2)$, we might wonder if we can make the set $X_1 \times X_2$ into a metric space. A natural approach might be to define a function $d : (X_1 \times X_2) \times (X_1 \times X_2) \to \R$ by 
\[d((x,y),(u,v)) = d_1(x,u)d_2(y,v)\]
for $(x,y)$ and $(u,v)$ in $X_1 \times X_2$. However, this $d$ does not define a metric. For example, if $x \in X_1$ and $y \neq v$ in $X_2$, then $d((x,y),(x,v)) = 0$ even though $(x,y) \neq (x,v)$. To make a metric, we can take a clue from the Euclidean metric on $\R \times \R$. On $\R$, the metric has the form $d_1(x,y) = |x-y|$, while on $\R^2$ the metric  is 
\[d_E((x_1,x_2), (y_1,y_2)) = \sqrt{ (x_1-y_1)^2 + (x_2-y_2)^2} = \sqrt{d_1(x_1,y_1)^2 + d_1(x_2,y_2)^2}.\]

So on $(X, d_1)$ and $(X_2 ,d_2)$ we could consider defining $d : (X_1 \times X_2) \times (X_1 \times X_2) \to \R$ by
\[d((x_1,x_2), (y_1,y_2)) = \sqrt{ d_1(x_1,y_1)^2 + d_2(x_2,y_2)^2}.\]

We show that $d$ is a metric If $x=(x_1,x_2)$ and $y=(y_1,y_2)$ are in $X_1 \times X_2$, then $d(x,y) = \sqrt{ d_1(x_1,y_1)^2 + d_2(x_2,y_2)^2}$ is nonnegative by definition. Also, the symmetry of $d_1$ and $d_2$ imply that $d(x,y) = d(y,x)$. Note that $d(x,y) = 0$ if and only if $d_1(x_1,y_1) = d_2(x_2,y_2) = 0$. But this happens if and only if $x_1=y_1$ and $x_2=y_2$, or if $x = y$. The last (and most difficult) property to verify is the triangle inequality. 

Let $z = (z_1, z_2)$ be in $X_1 \times X_2$. Then
\begin{align*}
d(x,z)^2 &=  d_1(x_1,z_1)^2 + d_2(x_2,z_2)^2 \\
	&\leq \left(d_1(x_1,y_1) + d_1(y_1,z_1)\right)^2 + \left(d_2(x_2,y_2) + d_2(y_2,z_2)\right)^2 \\
	&= \left(d_1(x_1,y_1)^2 + d_2(x_2,y_2)^2\right) + 2\left(d_1(x_1,y_1)d_1(y_1,z_1) + d_2(x_2,y_2)d_2(y_2,z_2)\right) \\
	&\qquad + \left(d_1(y_1,z_1)^2 + d_2(y_2,z_2)^2\right)  \\
	&= d(x,y)^2 + 2\left(d_1(x_1,y_1)d_1(y_1,z_1) + d_2(x_2,y_2)d_2(y_2,z_2)\right) + d(y,z)^2 \\
	&\leq d(x,y)^2 + d(y,z)^2.
\end{align*}
Since all terms are non-negative we conclude that 
\begin{align*}
d(x,z) \leq \sqrt{d(x,y)^2 + d(y,z)^2} &\leq \sqrt{d(x,y)^2 + 2 d(x,y)d(y,z) + d(y,z)^2} \\
	&= \sqrt{\left(d(x,y)+d(y,z)\right)^2} \\
	&= d(x,y) + d(y,z).
\end{align*}

In the next activity we consider products of open balls and open sets in products oif metric spaces.

\begin{activity} \label{act:ms_product_open} Let $X_1 = [1,2]$ and $X_2 = [3,4]$ as subspaces of $\R^2$ using the Euclidean metric. 
\ba
\item Explain in detail what the product space $X_1 \times X_2$ looks like. 

\item If $B_1$ is an open ball in $X_1$ and $B_2$ is an open ball in $X_2$, is $B_1 \times B_2$ an open ball in $X_1 \times X_2$? Explain.

\item If $B_1$ is an open ball in $X_1$ and $B_2$ is an open ball in $X_2$, is $B_1 \times B_2$ an open set in $X_1 \times X_2$? Explain.

\item Find, if possible, an open subset of $X_1 \times X_2$ that is not of the form $O_1 \times O_2$ where $O_1$ is open in $X_1$ and $O_2$ is open in $X_2$.

\ea

\end{activity}

\begin{comment}

\ActivitySolution

\ba
\item The points of the form $(x_1,x_2)$ with $1 \leq x_1 \leq 2$ and $3 \leq x_2 \leq 4$ is the square in $\R^2$ with sides of length $1$ parallel to the coordinate axes and center at $(1.5,3.5)$. 

\item Let $B_1 = B(a,r)$ be an open ball in $X_1$ and $B_2 = B(b,s)$ be an open ball in $X_2$. Then $B_1 = (a-r,a+r)$ and $B_2 = (b-s,b+s)$. The set $B_1 \times B_2$ is the rectangle $R = \{(x,y) \mid a-r<x<a+r, b-s<y<b+s\}$ in $\R^2$. But an open ball using the Euclidean metric is an open disk, not a rectangle. 

\item Let $B_1 = (a-r,a+r)$ be an open ball in $X_1$ and $B_2 = (b-s,b+s)$ be an open ball in $X_2$. Let $(u,v) \in B_1 \times B_2$. Then $u \in B_1$ and $v \in B_2$. Let $\epsilon = \min\{r-d_1(a,u), s-d_2(b,v)\}$. Let $(w,z) \in B((u,v),\epsilon)$. Then 
\[d_1(a,w) \leq d_1(a,u) + d_1(u,w) < d_1(a,u) + (r-d_1(a,u)) = r\]
and
\[d_2(b,z) \leq d_2(b,v) + d_2(v,z) < d_2(b,v) + (s-d_2(b,v)) = s.\]
So $B((u,v),\epsilon) \subseteq (B_1 \times B_2)$. This makes $B_1 \times B_2$ a neighborhood of each of its points and so $B_1 \times B_2$ is an open set. 

\ea

\end{comment}

Activity \ref{act:ms_product_open} shows that open sets in a product are more complicated than just products of open sets in the factors. We will return to product later when we consider topological spaces. 

We conclude with one final comment about products. We can make the Cartesian product of any number of metric spaces into a metric space with the same construction we used for the product of two spaces. 

\begin{definition} Let $(X_i, d_i)$ be metric spaces for $i$ from $1$ to some positive integer $n$. The \textbf{product metric space}\index{product metric space} $(X,d)$ is the Cartesian product
\[X = X_1 \times X_2 \times \cdots \times X_n = \prod_{i=1}^n X_i\]
with metric $d$ defined by 
\[d(x,y) = \sqrt{\sum_{i=1}^n d_i(x_i,y_i)^2}\]
when $x = (x_1, x_2, \ldots, x_n)$ and $y = (y_1, y_2, \ldots, y_n)$ are in $X$.
\end{definition}

The metric $d$ is called the \emph{product metric}\index{product metric} and the spaces $(X_i,d_i)$ are called the \emph{coordinate}\index{coordinate space} or \emph{factor}\index{factor space} spaces of $(X,d)$. The proof that $d$ is a metric is essentially the same as in the $n=2$ case, and is left to Exercise (\ref{ex:prod_metric}). 

\csection{Summary}
Important ideas that we discussed in this section include the following.
\begin{itemize}
\item A subset $A$ of a metric space $(X,d)$ is a metric space, called a subspace, by using the metric $d|_{A \times A}$ on $A$. 
\item If $X$ is a metric space and $A$ is a subspace of $X$, a subset $O_A$ of $A$ is open in $A$ if and only if $O_A = X \cap O$ for some open set $O$ in $X$. A subset $C_A$ of $A$ is closed in $A$ if $C_A = A \cap O_A$ for open set $O_A$ in $A$. Alternatively, a set $C_A$ is closed in $A$ if $C_A = A \cap C$ for some closed set $C$ in $X$.
\item Let $(X_i, d_i)$ be metric spaces for $i$ from $1$ to some positive integer $n$. The product metric space $(X,d)$ is the Cartesian product
\[X = X_1 \times X_2 \times \cdots \times X_n = \prod_{i=1}^n X_i\]
with metric $d$ defined by 
\[d(x,y) = \sqrt{\sum_{i=1}^n d_i(x_i,y_i)^2}\]
when $x = (x_1, x_2, \ldots, x_n)$ and $y = (y_1, y_2, \ldots, y_n)$ are in $X$.
\end{itemize}

\csection{Exercises}

\be

\item Determine if the following sets $S$ are open in the subspace $A$ of the topological space $(\R, d_E)$.

\ba

\item $S = [1,2)$ in $A = [1,3]$

\item $S = \{1, 2\}$ in $A = \Q$

\item $S = \{1,2\}$ in $A = \Z$

\ea

\begin{comment}

\ExerciseSolution

\ba

\item Since $S = A \cap (0,2)$, it follows that $S$ is open in $A$. 

\item Any open set in $\R$ that contains 1 also contains $B(1,r)$ for some $r > 0$. But this open ball contains points in $\Q$ that are not in $S$. We conclude that $S$ is not open in $A$. 

\item Since $S = \Z \cap (0,1.5) \cup (1.5,2.5)$, we conclude that $S$ is open in $A$.

\ea


\end{comment}

\item Let $O$ be an open set in a metric space $(X,d)$. Show that a subset $U$ of $O$ is open in $(O, d|_O)$ if and only if $U$ is open in $(X,d)$. 

\begin{comment}

\ExerciseSolution Let $U$ be an open set in $O$. Then $U = O \cap V$ for some open set $V$ in $X$ by definition. Since $O$ is open in $X$, it follows that $O \cap V$ is open in $X$. Thus, $U$ is open in $X$.

Conversely, suppose that $U \subset O$ and that $U$ is open in $X$. Then $U = O \cap U$ and $U$ is open in $O$. 

\end{comment}

\item Let $(X,d_X)$ and $(Y, d_Y)$ be metric spaces, and let $f : X \to Y$ be a continuous function. If $A$ is a subspace of $X$, must the restriction $f|_A$ of $f$ to $A$ mapping $A$ to $Y$ be continuous? Give a proof that the restriction is continuous, or an example to show that the restriction need not be continuous.

\begin{comment}

\ExerciseSolution Let $(X,d_X)$ and $(Y, d_Y)$ be metric spaces, and let $f : X \to Y$ be a continuous function. Let $A$ be a subspace of $X$. We will prove that $f|_A : A \to Y$ is continuous. Let $O$ be an open subset of $Y$. Since $f$ is continuous, we know that $f^{-1}(O)$ is open in $X$. We will demonstrate that $(f|_A)^{-1}(O) = A \cap f^{-1}(O)$. This will prove that $(f|_A)^{-1}(O)$ is an open subset of $A$ and that $f|_A$ is continuous. 

Let $x \in (f|_A)^{-1}(O)$. Then $x \in A$ and $f|_A(x) \in O$. But $f|_A(x) = f(x)$, so $f(x) \in O$ and $x \in f^{-1}(O)$. So $x \in A \cap f^{-1}(O)$ and $(f|_A)^{-1}(O) \subseteq A \cap f^{-1}(O)$. 

Now suppose that $x \in A \cap f^{-1}(O)$. Then $x \in A$ and $x \in f^{-1}(O)$. So $f(x) \in O$ and $x \in A$. Thus, $f|_A(x) \in O$ and $x \in (f|_A)^{-1}(O)$. It follows that $A \cap f^{-1}(O) \subseteq (f|_A)^{-1}(O)$. The two containments show that $A \cap f^{-1}(O) = (f|_A)^{-1}(O)$.

\end{comment}

\item \label{ex:relatively_closed_ms} Prove Theorem \ref{thm:relatively_closed_ms}. That is, let $(X,d)$ be a metric space and $A$ a subset of $X$. Prove that a subset $C_A$ of $A$ is closed in $A$ if and only if there is a closed set $C_X$ in $X$ so that $C_A = C_X \cap A$. 

\begin{comment}

\ExerciseSolution Let $C_A$ be a closed subset of $A$. Then $C_A = A \setminus O_A$ for some open set $O_A$ in $A$. Since $O_A$ is open in $A$, there exists an open set $O_X$ in $X$ such that $O_A = A \cap O_X$. Then $C_X = X \setminus O_X$ is a closed set in $X$. We will show that $C_A = A \cap C_X$. 

Let $c \in C_A$. Then $c \in A$ and $c \notin O_A$. Since $c \in A$, we also have $c \in X$. Now $c \in A$ but not in $O_A$, so $c$ cannot be in $O_X$. Thus, $c \in (X \setminus O_X) = C_X$. The fact that $c \in A$ implies that $c \in (A \cap C_X)$. 

Now suppose that $c \in (A \cap C_X)$. Then $c \in A$ and $c \in C_X$. Since $c \in C_X$, it follows that $c \in (X \setminus O_X)$. Thus, $c \notin O_X$. But $O_A \subseteq O_X$, so $c \notin O_A$. We conclude that $c \in (A \setminus O_A) = C_A$.  Therefore, $C_A = A \cap C_X$. 

Now suppose that $C_A = A \cap C_X$ for some closed subset $C_X$ of $X$. We will show that $C_A$ is closed in $A$ by showing that $C_A = A \setminus O_A$ for some open set $O_A$ in $A$. The fact that $C_X$ is closed in $X$ means that $O_X = X \setminus C_X$ is an open set in $X$. Then $O_A = A \cap O_X$ is an open set in $A$. We demonstrate that $C_A = A \setminus O_A$. 

Let $c \in C_A$. Then $c \in (A \cap C_X)$. So $c \in A$ and $c \in C_X$. Thus, $c \notin (X \setminus C_X) = O_X$. It follows that $c \notin O_A$. So $c \in A$ and $c \notin O_A$, which means that $c \in (A \setminus O_A)$. 

Finally, suppose that $c \in (A \setminus O_A)$. Then $c \in A$ and $c \notin O_A$. Since $c \notin O_A$, it follows that $c \notin O_X$. So $c \in (X \setminus O_X) = C_X$ and $c \in (A \cap C_X) = C_A$. We conclude that $C_A = A \setminus O_A$.


\end{comment}

\item Let $(X,d_X)$ and $(Y,d_Y)$ be metric spaces. Prove or disprove: the function $d: X \times Y \to \R$ defined by 
\[d((x_1,y_1), (x_2,y_2)) = d_X(x_1,x_2) + d_Y(y_1,y_2)\]
is a metric on $X \times Y$. 

\begin{comment}

\ExerciseSolution We will show that $d$ is a metric on $X \times Y$. Let $x = (x_1,y_1)$ and $y = (x_2,y_2)$ be in $X \times Y$. First note that 
\[d(x,y) = d_X(x_1,x_2) + d_Y(y_1,y_2) \geq 0 + 0 = 0\]
Also
\[d(x,y) = d_X(x_1,x_2) + d_Y(y_1,y_2) = d_X(x_2,x_1) + d_Y(y_2,y_1) = d(y,z).\]

Suppose $x = y$. Then $x_1=x_2$ and $y_1 = y_2$, so 
\[d(x,y) = d_X(x_1,x_2) + d_Y(y_1,y_2) = 0 + 0 = 0.\]
Conversely, suppose that $d(x,y) = 0$. Then $d_X(x_1,x_2) + d_Y(y_1,y_2) = 0$. But both $d_X(x_1,x_2)$ and $d_Y(y_1,y_2)$ are non-negative, which implies that $d_X(x_1,x_2 = 0 = d_Y(y_1,y_2)$. So $x_1=x_2$ and $y_1=y_2$, or $x = y$. 

Finally, let $z = (x_3,y_3) \in X \times Y$. Then 
\begin{align*}
d(x,z) &= d_X(x_1,x_3) + d_Y(y_1,y_3) \\
	&\leq (d_X(x_1,x_2) + d_X(x_2,x_3)) + (d_Y(y_1,y_2) + d_Y(y_2,y_3)) \\
	&= (d_X(x_1,x_2)+d_Y(y_1,y_2)) + (d_X(x_2,x_3) + d_Y(y_2,y_3)) \\
	&= d(x,y) + d(y,z).
\end{align*}

\end{comment}


\item \label{ex:prod_metric} Let $(X_i, d_i)$ be metric spaces for $i$ from $1$ to some positive integer $n$. Let $d: \prod_{i=1}^n X_i \to \R$ be defined 
\[d(x,y) = \sqrt{\sum_{i=1}^n d_i(x_i,y_i)^2}\]
when $x = (x_1, x_2, \ldots, x_n)$ and $y = (y_1, y_2, \ldots, y_n)$ are in $X$. Show that $d$ is a metric on $\prod_{i=1}^n X_i$.

\begin{comment}

\ExerciseSolution We show that $d$ is a metric If $x=(x_n)$ and $y=(y_n)$ are in $\prod_{i=1}^n X_i$, then $d(x,y) = \sqrt{ \sum_{i=1}^n d_i(x_i,y_i)^2 }$ is nonnegative by definition. Also, the symmetry of each $d_i$ implies that 
\[d(x,y) = \sqrt{ \sum_{i=1}^n d_i(x_i,y_i)^2 } = \sqrt{ \sum_{i=1}^n d_i(y_i,x_i)^2 } = d(y,x).\]
Note that $d(x,y) = 0$ if and only if $d_i(x_i,y_i) = 0$ for each $i$. But then $x_i=y_i$ for each $i$ and so $x = y$. 

Finally, let $z = (z_n)$ be in $\prod_{i=1}^n X_i$. Then
\begin{align*}
d(x,z)^2 &=  \sum_{i=1}^n d_i(x_i,z_i)^2  \\
	&\leq \sum_{i=1}^n \left(d_i(x_i,y_i) + d_i(y_i,z_i)\right)^2 \\
	&= \sum_{i=1}^n \left(d_i(x_i,y_i)^2 + 2d_i(x_i,y_i) d_i(y_i,z_i) + d_i(y_i,z_i)^2\right) \\
	&= \sum_{i=1}^n d_i(x_i,y_i)^2 + 2 \sum_{i=1}^n d_i(x_i,y_i) d_i(y_i,z_i) +  \sum_{i=1}^n d_i(y_i,z_i)^2 \\
	&= d(x,y)^2 + + 2 \sum_{i=1}^n d_i(x_i,y_i) d_i(y_i,z_i)  + d(y,z)^2 \\
	&\leq d(x,y)^2 + d(y,z)^2.
\end{align*}
Since all terms are non-negative, we conclude that 
\[d(x,z) \leq \sqrt{d(x,y)^2 + d(y,z)^2} \leq d(x,y) + d(y,z).\]

\end{comment}

\item \label{ex:Subspace_monotone_convergence} Let $(x_n)$ be a non-decreasing sequence of real numbers that is bounded above. That is, $x_n \leq x_{n+1}$ for every $n$ and there is a positive real number $K$ such that $x_n \leq K$ for every $n$. Show that the sequence $(x_n)$ converges. 

\begin{comment}

\ExerciseSolution Since the set $S = \{x_n \mid n \in \Z^+\}$ is bounded above, the set has a least upper bound $M$. We will show that $\lim x_n = M$. Let $\epsilon$ be a positive real number. If there is no number $N \in \Z^+$ such that $x_N > M - \epsilon$, then $M - \epsilon$ is an upper bound for $S$. But this contradicts the fact that $M$ is the least upper bound of $S$. So there is a positive integer $N$ such that $x_N > M - \epsilon$. Since the sequence $(x_n)$ is non-decreasing, it follows that $x_n > M - \epsilon$ for every $n \geq N$. Since $x_n \leq M$ as well for all $n \in \Z^+$ it follows that for $n \geq N$ we have 
\[d_E(x_n,M) < \epsilon.\]
Thus, $(x_n)$ converges to $M$.  

\end{comment}

\item It is possible to consider infinite products as metric spaces. One important example is a Hilbert space\index{Hilbert space} $H$, which consists of all infinite sequences $(x_n)$ where $x_n \in \R$ for every $n$ and $\sum_{k = 1}^{\infty} x_k^2$ is finite. Hilbert space has important applications in physics, particularly in quantum mechanics. 
\ba
\item Give two distinct elements in $H$ and one infinite sequence that is not in $H$. Explain your examples. 

\item We define the norm of an element $x = (x_n)$ in $H$ as 
\[\lVert x\rVert = \sqrt{\sum_{k=1}^{\infty} x_k^2}.\]
From this norm we can define a distance between elements $x = (x_n)$ and $y = (y_n)$ in $H$ as follows:
\[d(x,y) = \lVert x-y \rVert,\]
where $x-y = (x_n-y_n)$. Another way to write $d$ is 
\[d(x,y) = \sqrt{\sum_{k=1}^{\infty} (x_k-y_k)^2}.\]
One potential problem with this function $d$ is that we need to know that if $x$ and $y$ are in $H$, then $x-y \in H$. That is, show that if $\sum_{k=1}^{\infty} x_k^2$ and $\sum_{k=1}^{\infty} y_k^2$ are finite, then $\sum_{k=1}^{\infty} (x_k-y_k)^2$ is also finite. (Hint: Consider a finite sum and use Exercise \ref{ex:Subspace_monotone_convergence}.)

\item Show that $d$ is a metric on $H$.

\item Let $E^m = \{(x_n) \in H \mid x_k = 0 \text{ for  } k > m\}$. Let $f: E^m \to \R^m$ be defined by $f((x_n)_{n=1}^{\infty}) = (x_n)_{n=1}^m$. Show that $f$ is a bijection such that $d((x_n), (y_n)) = d_E(f((x_n)), f((y_n)))$ for any elements $(x_n)$, $(y_n)$ in $H$. So $E^m$ is essentially the same as $\R^m$ and so we can consider the space $\R^m$ as embedded in $H$ as a subspace of $H$ for every $m \in \Z^+$. 
 
\ea

\begin{comment}

\ExerciseSolution

\ba

\item Since the sequence $\sum_{k=1}^{\infty} \frac{1}{k^2}$ is a $p$-series with $p = 2 > 1$, we know that  $\sum_{k=1}^{\infty} \frac{1}{k^2}$ converges. Thus, the sequence $\left(\frac{1}{n}\right)$ is in $H$. Similarly, the sequence $\left(\frac{1}{n^2}\right)$ is in $H$.  The the sequence $\sum_{k=1}^{\infty} \frac{1}{k}$ is a $p$-series with $p = 1$, which diverges. Therefore, the sequence $\left(\frac{1}{\sqrt{n}}\right)$ is not in $H$.

\item Expanding $(x_k-y_k)^2$ shows that 
\begin{align*}
\sum_{k=1}^{n} (x_k-y_k)^2 &= \sum_{k=1}^{n} x_k^2 - 2\sum_{k=1}^{n} x_ky_k + \sum_{k=1}^{n} y_k^2 \\
	&\leq \sum_{k=1}^{n} x_k^2 + 2\sum_{k=1}^{n} lx|_k|y|_k + \sum_{k=1}^{n} y_k^2 \\
	&\leq \sum_{k=1}^{n} x_k^2 + 2\left(\sqrt{\sum_{k=1}^n x_k^2}\right) \left(\sqrt{\sum_{k=1}^n y_k^2}\right)+ \sum_{k=1}^{n} y_k^2.
\end{align*}
Now $\sum_{k=1}^n x_k^2 \leq \lVert x \rVert^2$ and $\sum_{k=1}^n y_k^2 \leq \lVert y \rVert^2$ for any $n$, so 
\[\sum_{k=1}^{n} (x_k-y_k)^2 \leq \lVert x \rVert^2 + 2 \lVert x \rVert \lVert y \rVert + \lVert y \rVert^2 = (\lVert x \rVert + \lVert y \rVert)^2.\]
So the sequence $\left(\sum_{k=1}^{n} (x_k-y_k)^2\right)$ is a non-decreasing sequence of real numbers that is bounded above. Hence, by Exercise \ref{ex:Subspace_monotone_convergence}, the sequence $\left(\sum_{k=1}^{n} (x_k-y_k)^2\right)$ converges and its limit is the real number $\sum_{k=1}^{\infty} (x_k-y_k)^2$. 

\item Let $x=(x_n)$, $y=(y_n)$, and $z = (z_n)$ be in $H$. By definition, 
\[d(x,y) = \lVert x-y \rVert \geq 0\]
and 
\[d(x,y) = \lVert x-y \rVert = \lVert y-x \rVert = d(y,x).\]
Note that 
\[d(x,y) = \sqrt{\sum_{k=1}^{\infty} (x_k-y_k)^2} = 0\]
if and only if $|x_k-y_k|=0$ for every $k$. But this happens if and only if $x = y$. 

Finally, 
\begin{align*}
d(x,z)^2 &=  \sum_{k=1}^{\infty} (x_k-z_k)^2   \\
	&= \sum_{k=1}^{\infty} (|x_k-z_k|)^2   \\
	&= \sum_{k=1}^{\infty} \left(|(x_k-y_k)+(y_k-z_k)|\right)^2 \\
	&\leq \sum_{k=1}^{\infty} \left(|x_k-y_k|+|y_k-z_k|\right)^2 \\
	&= \sum_{k=1}^{\infty} |x_k-y_k|^2 + 2|x_k-y_k| \ |y_k-z_k|+|y_k-z_k|^2 \\
	&= \sum_{k=1}^{\infty} (x_k-y_k)^2 + 2\sum_{k=1}^{\infty} |x_k-y_k| \ |y_k-z_k|+ \sum_{k=1}^{\infty} (y_k-z_k)^2 \\
	&\leq \sum_{k=1}^{\infty} (x_k-y_k)^2 +  \sum_{k=1}^{\infty} (y_k-z_k)^2 \\
	&\leq d(x,y)^2 + d(y,z)^2.
\end{align*}
Since all terms are non-negative we conclude that 
\[d(x,z) \leq \sqrt{d(x,y)^2 + d(y,z)^2} \leq d(x,y) + d(y,z).\]

\item Let $x = (x_n)$ and $y = (y_n)$ be in $E^m$ and assume that $f((x_n)) = f((y_n))$. Since $x$ and $y$ are in $E^m$, we know that $x_k = y_k = 0$ for $k > m$. The fact that $f((x_n)) = f((y_n))$ implies that $(x_n)_{n=1}^m = (y_n)_{n=1}^m$, so $x_n = y_n$ for $1 \leq n \leq m$. We conclude that $x=y$ and that $f$ is an injection. Now let $(z_n)_{n=1}^m$ be in $\R^m$ and let $z = (z_n)_{n=1}^{\infty}$ where $z_k = 0$ for $k > m$. Since there are only a finite number of non-zero terms in $z$, we know that $z \in H$. Thus, $z \in E^m$ and $f(z) = (z_n)_{n=1}^m$. Therefore, $f$ is a surjection and a bijection.

Using the fact that $x_k = y_k =0$ when $k > m$ we also see that 
\begin{align*}
d((x_n), (y_n)) &= \sum_{k=1}^{\infty} (x_k-z_k)^2  \\
	&= \sum_{k=1}^{m} (x_k-z_k)^2  \\
	&= d_E((x_n)_{n=1}^m, (y_n)_{n=1}^m) \\
	&= d_E(f((x_n)), f((y_n))).
\end{align*}
 
\ea

\end{comment}


\item For each of the following, answer true if the statement is always true. If the statement is only sometimes true or never true, answer false and provide a concrete example to illustrate that the statement is false. If a statement is true, explain why. 
	\ba
	\item If $d$ is the discrete metric on a metric space $X$, then for any subspace $A$ of $X$, the restriction of $d$ to $A$ is the discrete metric.
	
	\item If $d$ is a metric on a space $X$ that is not the discrete metric, and if $A$ is a subset of $X$, then $d|_A$ cannot be the discrete metric. 
	
	\item Let $A$ be a subspace of a metric space $(X,d)$. If a sequence $(a_n)$ is in $A$ and $\lim a_n = a$ for some $a \in A$, then $\lim a_n = a$ in $X$. 

	\item Let $A$ be a subspace of a metric space $(X,d)$. If a sequence $(a_n)$ is in $A$ and $\lim a_n = a$ for some $a \in X$, then $\lim a_n = a$ in $A$. 	
	
	\item If $(X,d_X)$ and $(Y, d_Y)$ are metric spaces, then the function $d: X \times Y \to \R$ defined by 
	\[d((x_1,y_1), (x_2,y_2)) = \max\{d_X(x_1,x_2), d_Y(y_1,y_2)\}\]
	is a metric on $X \times Y$. 
	
	\item If $(X,d_X)$ and $(Y, d_Y)$ are metric spaces, then the function $d: X \times Y \to \R$ defined by 
	\[d((x_1,y_1), (x_2,y_2)) = d_X(x_1,x_2)d_Y(y_1,y_2)\]
	is a metric on $X \times Y$.
		
	\ea

\begin{comment}

\ExerciseSolution

	\ba
	\item This statement is true. If $d$ is the discrete metric on a metric space $X$ and if $A$ is a subset of $X$, then $d|_A(a,b)$ still equals $1$ if $a \neq b$ and $0$ if $a = b$ because $a$ and $b$ are elements of $X$. 
		
	\item This statement is false. Let $X = \R$ with the Euclidean metric. If $A = \{0,1\}$, then $d|A(0,1) = 1$ while $d|A(0,0) = d|A(1,1) = 0$. 
	
	\item This statement is true since all $a_n$ and $a$ are also in $X$. 

	\item This statement is false. Consider the sequence $(a_n)$ where $a_n = \left(1+\frac{1}{n}\right)^n$. Then $a_n \in \Q$ for every $n$ and $\lim a_n = e$ in $\R$. However, since $e \notin \Q$, the sequence $(a_n)$ has no limit in $\Q$.  	
	
	\item This statement is true. Let $x = (x_1,y_1)$, $y = (x_2,y_2)$, and $z = (x_3,y_3)$ in $X \times Y$. Since $d_X(x_1,x_2) \geq 0$ and $d_Y(y_1,y_2) \geq 0$, we know that $d(x,y) \geq 0$. Also,
\[d(x,y) = \max\{d_X(x_1,x_2), d_Y(y_1,y_2)\} = \max\{d_Y(y_1,y_2), d_X(x_1,x_2)\} = d(y,x).\]	

If $x = y$, then $d_X(x_1,x_2) 0 = d_Y(y_1,y_2)$ and so $d(x,y) = 0$. Conversely, if $d(x,y) = 0$, it must be the case that $d_X(x_1,x_2) 0 = d_Y(y_1,y_2)$. Thus, $x = y$. 

Finally, 
\begin{align*}
d(x,z) &= \max\{d_X(x_1,x_3), d_Y(y_1,y_3)\} \\
	&\leq \max\{d_X(x_1,x_2)+d_X(x_2,x_3), d_Y(y_1,y_2)+d_Y(y_2,y_3)\} \\	
	&\leq \max\{d_X(x_1,x_2), d_Y(y_1,y_2)\} + \max\{d_X(x_2,x_3), d_Y(y_2,y_3)\} \\
	&= d(x,y) + d(y,z).
\end{align*}
	
	\item This statement is false. Let $(X,d_X) = (Y, d_Y) = (\R, d_E)$, and let $x = (0,0)$ and $y = (0,1)$. Then $x \neq y$ but 	\[d(x,y) = d_X(0,0)d_Y(0,1) = 0.\]
		
	\ea



\end{comment}

\ee

