\achapter{1}{Hausdorff Topological Spaces}\label{sec:Hausdorff_topology}


\vspace*{-17 pt}
\framebox{\hspace*{3 pt}
\parbox{4.7 in}{\begin{fqs}
\item 

\end{fqs}} \hspace*{3 pt}}

\vspace*{13 pt}

\csection{Introduction}

Our intuition usually serves us well when working in $\R^n$ with the Euclidean metric. The ideas of open and closed sets correspond to exactly those sets we think should be open and closed. We must be careful, though, when working in topological spaces as counterintuitive things can happen. Most sets in which we work exhibit expected behavior, so we identify those characterize that we consider to be ``normal" and then characterize the spaces that behave in these ``normal" ways. One such family of spaces are the Hausdorff spaces.

\begin{pa} ~
\be
\item  In $\R^n$ with the Euclidean metric, every single element set is closed. Does this property hold in the topological space $(X, \tau)$, where $X = \{a, b, c\}$ and $\tau = \{\emptyset, \{a\}, \{a, b\}, \{a, c\}, X\}$? Explain. 

One property of closed sets in metric spaces was that closed sets contained limits of convergent sequences. We defined a sequence in a metric space as a function from the positive integers to the space, and we can apply the same definition in topological spaces.

HAVE WE DONE THIS ALREADY?

\begin{definition} A \textbf{sequence} in a topological space $X$ is a function $f : \Z^+ \to X$.
\end{definition}

We use the same notation and terminology related to sequences as we did in metric spaces: we will write $(x_n)$ to represent a sequence $f$, where $x_n = f(n)$ for each $n \in \Z^+$. In addition, we defined convergence of sequences in a metric space, and were able to phrase this concept in terms of open sets. Consequently, we can make a similar definition in topological spaces. As we will see, though, the notion of convergent sequence is less useful in general topological spaces than in metric spaces.
\begin{definition}  A sequence $(x_n)$ in a topological space $X$ converges to the point $x \in X$ if, for each open set $O$ that contains $x$ there exists a positive integer $N$ such that $x_n \in O$ for all $n \geq N$.
\end{definition}

If a sequence $(x_n)$ converges to a point $x$, we call $x$ a limit of the sequence $(x_n)$.

\item In metric spaces, limits of convergent sequences are unique. Does this property hold in the topological space $(X,\tau)$, where $X = \{a, b, c\}$ and $\tau = \{\emptyset, \{a\}, \{a b\}, \{a, c\}, X\}$? Explain. (Hint: Examine constant sequences.)

%Solution. No. Consider the constant sequence$(a)$. The neighborhoods of $b$ are $\{a, b\}$ and $X$. Each of these neighborhoods contains the entire sequence $(a)$, so $b$ is a limit of the sequence $(a)$. Similarly, the neighborhoods of $c$ are $\{a, c\}$ and $X$. Each of these neighborhoods contains the entire sequence $(a)$, so $c$ is a limit of the sequence $(a)$.

Spaces in which single point sets are not closed, or in which sequences can converge to more than one point, do not generally occur in applications. Also, this type of behavior limits the results that one can prove about such spaces. As a result, we define classes of topological spaces whose behaviors are closer to what our intuition suggests. One such class is the following.

\begin{definition} A topological space $X$ is a \textbf{Hausdorff}\index{Hausdoff space} space if for each pair $x$, $y$ of distinct points in $X$, there exists open sets $O_x$ of $x$ and $O_y$ of $y$ such that $O_x \cap O_y = \emptyset$. 
\end{definition}
In other words, a topological space is Hausdorff if we can separate distinct points with disjoint open sets.

\item 
	\ba
	\item Let $(X, d)$ be a metric space and let $x$ and $y$ be distinct elements of $X$. Is it possible to find disjoint open balls $B_x$ and $B_y$ centered at $x$ and $y$, respectively? Why or why not? Is every metric space Hausdorff?
	
%Solution. Let $m = d(x,y)$. Then $B(x,m)$ and $B(y,m)$ are open sets that separate $x$ and $y$. So every metric space is Hausdorff.
	
	\item Consider $(\R, \tau_{FC})$, where $\tau_{FC}$ is the finite complement topology. Let $x$ and $y$ be distinct elements of $\R$. Suppose $O_x$ is an open set in $\R$ that contains $x$ and $O_y$ an open set in $\R$ that contains $y$. What can we say about $O_x \cap O_y$? Explain. Is $(\R, \tau_{FC})$ Hausdorff?
	
Solution. We know that $O_x \cap O_y$ is an open set, so $\R \setminus (O_x \cap O_y)$ is finite. Therefore, $O_x \cap O_y \neq \emptyset$. It follows that $(\R, \tau_{FC})$ is not Hausdorff.
		\ea
\ee

\end{pa}


\begin{comment}

\ActivitySolution

\be
\item  In $\R^n$ with the Euclidean metric, every single element set is closed. Does this property hold in the topological space $(X, \tau)$, where $X = \{a, b, c\}$ and $\tau = \{\emptyset, \{a\}, \{a, b\}, \{a, c\}, X\}$? Explain. 

%Solution. No, the set $X \setminus \{a\} = \{b, c\}$ is not open, so $\{a\}$ is not closed.

One property of closed sets in metric spaces was that closed sets contained limits of convergent sequences. We defined a sequence in a metric space as a function from the positive integers to the space, and we can apply the same definition in topological spaces.

\begin{definition} A \textbf{sequence} in a topological space $X$ is a function $f : \Z^+ \to X$.
\end{definition}

We use the same notation and terminology related to sequences as we did in metric spaces: we will write $(x_n)$ to represent a sequence $f$, where $x_n = f(n)$ for each $n \in \Z^+$. In addition, we defined convergence of sequences in a metric space, and were able to phrase this concept in terms of open sets. Consequently, we can make a similar definition in topological spaces. As we will see, though, the notion of convergent sequence is less useful in general topological spaces than in metric spaces.
\begin{definition}  A sequence $(x_n)$ in a topological space $X$ converges to the point $x \in X$ if, for each open set $O$ that contains $x$ there exists a positive integer $N$ such that $x_n \in O$ for all $n \geq N$.
\end{definition}

If a sequence $(x_n)$ converges to a point $x$, we call $x$ a limit of the sequence $(x_n)$.

\item In metric spaces, limits of convergent sequences are unique. Does this property hold in the topological space $(X,\tau)$, where $X = \{a, b, c\}$ and $\tau = \{\emptyset, \{a\}, \{a b\}, \{a, c\}, X\}$? Explain. (Hint: Examine constant sequences.)

%Solution. No. Consider the constant sequence$(a)$. The neighborhoods of $b$ are $\{a, b\}$ and $X$. Each of these neighborhoods contains the entire sequence $(a)$, so $b$ is a limit of the sequence $(a)$. Similarly, the neighborhoods of $c$ are $\{a, c\}$ and $X$. Each of these neighborhoods contains the entire sequence $(a)$, so $c$ is a limit of the sequence $(a)$.

Spaces in which single point sets are not closed, or in which sequences can converge to more than one point, do not generally occur in applications. Also, this type of behavior limits the results that one can prove about such spaces. As a result, we define classes of topological spaces whose behaviors are closer to what our intuition suggests. One such class is the following.

\begin{definition} A topological space $X$ is a \textbf{Hausdorff} space if for each pair $x$, $y$ of distinct points in $X$, there exists open sets $O_x$ of $x$ and $O_y$ of $y$ such that $O_x \cap O_y = \emptyset$. 
\end{definition}
In other words, a topological space is Hausdorff if we can separate distinct points with disjoint open sets.

\item 
	\ba
	\item Let $(X, d)$ be a metric space and let $x$ and $y$ be distinct elements of $X$. Is it possible to find disjoint open balls $B_x$ and $B_y$ centered at $x$ and $y$, respectively? Why or why not? Is every metric space Hausdorff?
	
%Solution. Let $m = d(x,y)$. Then $B(x,m)$ and $B(y,m)$ are open sets that separate $x$ and $y$. So every metric space is Hausdorff.
	
	\item Consider $(\R, \tau_{FC})$, where $\tau_{FC}$ is the finite complement topology. Let $x$ and $y$ be distinct elements of $\R$. Suppose $O_x$ is an open set in $\R$ that contains $x$ and $O_y$ an open set in $\R$ that contains $y$. What can we say about $O_x \cap O_y$? Explain. Is $(\R, \tau_{FC})$ Hausdorff?
	
Solution. We know that $O_x \cap O_y$ is an open set, so $\R \setminus (O_x \cap O_y)$ is finite. Therefore, $O_x \cap O_y \neq \emptyset$. It follows that $(\R, \tau_{FC})$ is not Hausdorff.
		\ea
\ee


\end{comment}


\csection{Introduction}

As we saw in our Preview Activity, unexpected behavior can happen in arbitrary topological spaces. As a result, we define certain classes of topological spaces whose behavior is in accord with what we expect. Hausdorff spaces are one such type of space. 



\begin{definition} A topological space $X$ is a \textbf{Hausdorff} space if for each pair $x$, $y$ of distinct points in $X$, there exists open sets  $O_x$ of $x$ and $O_y$ of $y$ such that $O_x \cap O_y = \emptyset$.
\end{definition}



%Examples of non-Hausdorff spaces are $\R$ with the finite complement topology and any set with more than one point using the trivial topology. 

Hausdorff spaces are important because they behave in some ways that our intuition expects spaces to behave. 



\begin{activity} \label{act:Hausdorff} Listed are two properties in metric spaces that do not occur in arbitrarily topological spaces. Do Hausdorff spaces have these properties? Explain. 
\ba
\item Single point sets are closed.



\item A sequence of points can have at most one limit.



\ea

\end{activity}



%\begin{theorem} Each single point subset of a Hausdorff topological space is closed.
%\end{theorem}

%\begin{proof} Let $X$ be a Hausdorff topological space, and let $A = \{a\}$ for some $a \in X$. To show that $A$ is closed, we prove that $X \setminus A$ is open. Let $x \in X \setminus A$. Then $x \neq a$. So there exist open sets $O_x$ of $x$ and $O_a$ of $a$ such that $O_x \cap O_a = \emptyset$. So $a \notin O_x$ and $O_x \subseteq X \setminus A$. Thus, every point of $X \setminus A$ is an interior point and $X \setminus A$ is an open set. This verifies that $A$ is a closed set.
%\end{proof}

%

%\begin{theorem} A sequence of points in a Hausdorff topological space can have at most one limit in the space.
%\end{theorem}

%\begin{proof} Let $X$ be a Hausdorff topological space, and let $(x_n)$ be a sequence in $X$. Suppose $(x_n)$ converges to $a \in X$ and to $b \in X$. Suppose $a \neq b$. Then there exist open sets $O_a$ of $a$ and $O_b$ of $b$ such that $O_a \cap O_b = \emptyset$. But the fact that $(x_n)$ converges to $a$ implies that there is a positive integer $N$ such that $x_n \in O_a$ for all $n \geq N$. But then $x_n \notin O_b$ for any $n \geq N$. This contradicts the fact that $(x_n)$ converges to $b$. We conclude that $a=b$ and that the sequence $(x_n)$ can have at most one limit in $X$. 
%\end{proof}

%

We often restrict ourselves to working in Hausdorff spaces to ensure that strange things like multiple limits of sequences cannot happen. 



 One final note about Hausdorff spaces. As one might expect, being Hausdorff is a topological property.
 


\begin{theorem} Let $(X, \tau_X)$ and $(Y, \tau_Y)$ be homeomorphic topological spaces. Then $X$ is a Hausdorff space if and only if $Y$ is a Hausdorff space. 
\end{theorem}

\begin{proof} Let $(X, \tau_X)$ and $(Y, \tau_Y)$ be topological spaces, and let $f : X \to Y$ be a homeomorphism. Assume that $X$ is a Hausdorff space. To show that $Y$ is a Hausdorff space, let $y_1$ and $y_2$ be distinct points in $Y$. Since $f$ is a surjection, there exist $x_1, x_2 \in X$ such that $f(x_1) = y_1$ and $f(x_2)=y_2$. The fact that $f$ is an injection means that $x_1 \neq x_2$. Given that $X$ is a Hausdorff space, there exist disjoint open sets $O_1$ of $x_1$ and $O_2$ of $x_2$ in $X$. Because $f^{-1}$ is continuous, the sets $M_1=f(O_1)$ and $M_2 = f(O_2)$ are open sets containing $y_1$ and $y_2$, respectively, in $Y$. Also,
\[M_1 \cap M_2 = f(O_1) \cap f(O_2) = f(O_1 \cap O_2) = \emptyset.\]
So $M_1$ and $M_2$ are disjoint open sets containing $y_1$ and $y_2$. We conclude that $Y$ is a Hausdorff space. 

The converse is true because $f^{-1}: Y \to X$ is a homeomorphism.
\end{proof}

