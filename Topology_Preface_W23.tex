\achapter{0}{Preface}\label{act:Preface}

\csection{A Free and Open-Source Linear Algebra Text} 
 
Mathematics is for everyone -- whether as a gateway to other fields or as background for higher level mathematics.  I made this textbook available to everyone for free download for their own non-commercial use. It could be especially useful for instructors who are looking for an inquiry-based textbook, as a supplemental resource to accompany their course, or for someone interested in learning some topology on their own. If an instructor would like to make changes to any of the files to better suit their students' needs, source files for the text are available by making a request to the author.

This work is licensed under the Creative Commons Attribution-NonCommercial-ShareAlike 4.0 International License.  The graphic 
\begin{center}
\includegraphics{CClicense.eps}
\end{center}
shows that the work is licensed with the Creative Commons, that the work may be used for free by any party so long as attribution is given to the author(s), that the work and its derivatives are used in the spirit of ``share and share alike,'' and that no party may sell this work or any of its derivatives for profit.  Full details may be found by visiting
\begin{center}
\href{http://creativecommons.org/licenses/by-nc-sa/3.0/}{\texttt{http://creativecommons.org/licenses/by-nc-sa/3.0/}}
\end{center} 
or sending a letter to Creative Commons, 444 Castro Street, Suite 900, Mountain View, California, 94041, USA. 

\csection{Goals} 
Over many years I have taught topology I developed pre- and in-class activities that I used to supplement the texts I had adopted. Eventually I had enough material to eliminate the reliance on outside texts and could rely on the activities. This book is built on those activities and is intended as a one semester introduction to point-set topology. The emphasis for this book is to have students be active learners and to help develop their intuition through working activities and examples. Although it is difficult to capture the essence of active learning in a textbook, this book is an attempt to do just that.

The goals for these materials are several.
\begin{itemize}
\item To carefully introduce the ideas behind the definitions and theorems to help students develop intuition and understand the logic behind them.
\item To help students understand that mathematics is not done as it is often presented. I expect students to experiment through examples, make conjectures, and then refine or prove their conjectures. I believe it is important for students to learn that definitions and theorems don't pop up completely formed in the minds of most mathematicians but are the result of much thought and work.
\item To help students develop their communication skills in mathematics. Students are expected to read and complete activities before class and come prepared with questions. Students regularly work in groups and present their work in class. Outside of class students work pre-class activities that are designed to help review previous material and to prepare them for the discussion of new material. In addition, students also individually write solutions to exercises on which they receive significant feedback. Communication skills are essential in any discipline and a heavy focus is placed on their development.
\end{itemize}


\csection{Layout}

Each section of the book contains preview activities, in-class activities, and exercises. The various types of activities serve different purposes.

\begin{itemize}
\item Preview activities are designed for students to complete before class to motivate the upcoming topic and prepare them with the background and information they need for the class activities and discussion.
\item The in-class activities engage students in common intellectual experiences. These activities are used to provide motivation for the material, opportunities for students to prove substantial course material on their own, or examples to help reinforce the meanings of definitions or theorems and their proofs. The ultimate goal is to help students develop their intuition for and understanding of abstract concepts. Students often complete these in-class activities, then present their results to the entire class.
\end{itemize}

Each section contains a collection of exercises. The exercises occur at a variety of levels of difficulty and most force students to extend their knowledge in different ways. While there are some standard, classic problems that are included in the exercises, many problems are open ended and expect a student to develop and then verify conjectures. 

\csection{Organization}
This text begins by formally introducing sets and functions. Although these topics are familiar to students, it is my experience that the level of understanding of sets and functions for most students is not yet sufficiently deep enough to ensure success with functions throughout the course. Chapter 2 focuses on metric spaces. It is my belief and my experience that students better understand the abstract concepts of neighborhoods, open sets, continuity, etc., if they are first experienced in a more familiar, concrete context like metric spaces. Metric spaces are easier for students to grasp than general topological spaces as they provide a notion of distance that is comforting to students. Metric spaces also allow one to introduce and motivate important topological concepts in a more familiar context. For example, by first encountering continuity of functions in a metric space setting, and revisiting the idea from different perspectives, the definition of continuity in the more abstract setting of topological spaces seems more accessible and natural. This perspective was also noted by Felix Hausdorff, who is considered as one of the founders of modern topology. His text \emph{Grundz\"{u}ge der Mengenlehre} (Fundamentals of Set Theory) (Felix Hausdorff, Leipzig, Von Veit, 1914. Translated to English as \emph{Set Theory} by John R. Aumann et al, New York, Chelsea PublishingCompany, 1957) provided one of the first systematic treatments of topological spaces. As Hausdorff writes (p. 210) concerning the introduction of the concepts of topology following topics of basic set theory, 
\begin{quote}
``A quite generally worded theory of this nature would of course cause considerable complications, and deliver few positive results. But among the special examples that occupy a heightened interest belongs, apart from the theory of a [totally] ordered sets, especially the theory of point-sets in space, in fact here the foundational relationship is again a function of pairs of elements, namely the distance between two points: a function which however now is capable of infinitely many values."
\end{quote}

\csection{To the Instructor}

While this text is intended as a one semester introduction to point-set topology, there is more material in the text than my students have been able to comfortably digest in a single semester. With that in mind, some care should be taken to ensure that what you feel are the most important topics are ones that are discussed. As mentioned earlier, I think students really benefit from a thorough review and discussion of sets and functions in the first two sections. From there, the sections build on each other. However, one can omit Section \ref{sec:metric_spaces_apps} on Applications of Metric Spaces without loss of continuity. My goal for a single semester course is to be sure that we can investigate compactness and connectedness. With this in mind, it is possible to omit Section \ref{sec:quotients} on quotient spaces if time is tight. 


\csection{Acknowledgments}

The Grand Valley Steve University libraries provided a grant to support my colleagues, John Golden and Clark Wells, to review a draft version of this text. Their thorough reading and comments on the draft have made for a much better book. I am also indebted to my students who have all been test subjects for earlier versions of this material. Without their hard work and suggestions, this book would not exist. 

\csection{To the Student}

I have endeavored to keep the prerequisite material to a minimum for this text. The first two sections discuss sets and functions. Most of this material should be review for the reader, but the importance of these ideas in the subsequent sections makes it a good idea to spend time on these sections. That being said, it will be very helpful for the reader to have some background in reading and writing mathematical proofs. 

The objectives of this book and its inquiry-based format place the responsibility of learning the material where it belongs -- on your shoulders. It is imperative that you engage in the material by completing the preview activities and the in-class activities in order to develop your intuition and understanding of the material. If you do this, and ask questions when you have them, your probability of success will be greatly enhanced. Good luck!



