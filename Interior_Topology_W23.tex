\achapter{12}{Interiors in Topological Spaces}\label{sec:Interiors_topology}


\vspace*{-17 pt}
\framebox{
\parbox{\dimexpr\linewidth-3\fboxsep-3\fboxrule}
{\begin{fqs}
\item 

\end{fqs}}}% \hspace*{3 pt}}

\vspace*{13 pt}

\csection{Introduction}
We have seen that topologies define open sets. As in metric spaces, open sets can be characterized in terms of their interior points. We defined interior points in metric spaces in terms of neighborhoods -- the same holds true in topological spaces. 

\begin{definition} Let $A$ be a subset of a topological space $X$. A point $a \in A$ is an \textbf{interior point}\index{interior point} of $A$ if $A$ is a neighborhood of $a$.
\end{definition}

Remember that a set is a neighborhood of a point if the set contains an open set that contains the point. By definition, every open set is a neighborhood of each of its points, so every point of an open set $O$ is an interior point of $O$. Conversely, if every point of a set $O$ is an interior point, then $O$ is a neighborhood of each of its points and is open. This argument is summarized in the next theorem. 

\begin{theorem} Let $X$ be a topological space. A subset $O$ of $X$ is open if and only if every point of $O$ is an interior point of $O$. 
\end{theorem}

The collection of interior points in a set form a subset of that set, called the \emph{interior} of the set.

\begin{definition} The \textbf{interior} of a subset $A$ of a topological space $X$ is the set
\[\Int(A) = \{a \in A \mid a \text{ is an interior point of } A\}.\]
\end{definition}

\begin{pa} 
\be
\item Consider $(\R, \tau)$, where $\tau$ is the standard topology. Let $A=(-\infty,0)\cup (1,2]\cup \{3\}$ in $\R$. What is $\Int(A)$? What is the largest open subset of $\R$ contained in $A$?

\item Consider $(\R, \tau)$, where $\tau$ is the discrete topology. Let $A=(-\infty,0) \cup (1,2] \cup \{3\}$ in $\R$. What is $\Int(A)$? What is the largest open subset of $\R$ contained in $A$?

\item Consider $(\R, \tau)$, where $\tau$ is the finite complement topology. Let $A=(-\infty,0) \cup (1,2] \cup \{3\}$ in $\R$. What is $\Int(A)$? What is the largest open subset of $\R$ contained in $A$?

\item Consider $(\R, \tau)$ where $\tau$ is the standard topology. Let $A = \Q$. What is $\Int(A)$. What is the largest open subset of $\R$ contained in $A$?

\item Let $X = \{a,b,c,d\}$ and let 
\[\tau = \{\emptyset, \{a\}, \{a,b\}, \{c\}, \{d\}, \{c,d\}, \{a,c,d\}, \{a,c\}, \{a,d\}, \{a,b,c,\}, \{a,b,d\}, X\}.\]
Assume that $\tau$ is a topology on $X$. Let $A = \{b,c,d\}$. What is $\Int(A)$? What is the largest open subset of $X$ contained in $A$?

\item Let $A, B$ be two subsets in a topological space $X$. What can you say about the relationships between $\Int(A\cap B), \Int(A\cup B)$ and $\Int(A)\cap \Int(B), \Int(A)\cup \Int(B)$, respectively? Verify your results.

\ee

\end{pa}

\begin{comment}

\ActivitySolution

\be
\item If $x \in (-\infty,0)$, then $B(x,|x|) \subseteq A$. If $x \in (1,2)$ and $r = \min\{x-1, 2-x\}$, then $B(x,r) \subseteq A$. So every point in $(-\infty,0) \cup (1,2)$ is an interior point of $A$. For any $\epsilon > 0$, the ball $B(2,\epsilon)$ contains points larger than $2$ and the ball $B(3,\epsilon)$ contains points larger than $3$. So neither $2$ nor $3$ is an interior point of $A$. This makes $\Int(A) = (-\infty,0) \cup (1,2)$. We also see that $\Int(A)$ is the largest open subset of $X$ contained in $A$.    

\item Since every subset of $\R$ is an open set, every set is a neighborhood of each of its points. That and the fact that $\Int(A) \subseteq A$ by definition allows us to conclude that $\Int(A) = A$. Since every set is open, the largest open set in $\R$ contained in $A$ is also $A$. 

\item Recall that a set $O$ is open in the finite complement topology if $\R \setminus O$ is finite. Since $\R \setminus A$ is not finite, no subset of $A$ is open and so no subset of $A$ can be a neighborhood of each of its points. It follows that $\Int(A) = \emptyset$. By the same reasoning, $A$ contains no open subset of $\R$, so the largest open subset of $\R$ contained in $A$ is $\emptyset$. 

\item Let $x \in A$ and let $\epsilon$ be greater than $0$. Then $B(x,\epsilon)$ contains irrational numbers. So no point of $A$ is an interior point of $\Q$. Thus, $\Int(\Q) = \emptyset$, which is also the largest open subset of $X$ contained in $\Q$. 

\item Since $\{c\} \subseteq A$ and $\{d\} \subseteq A$, we see that $c$ and $d$ are interior points of $A$. However, there is no open subset of $A$ that contains $b$, so $b$ is not an interior point of $A$. It follows that $\Int(A) = \{c,d\}$, which is also the largest open subset of $X$ contained in $A$. 

\item Let $X = \R$ with the standard topology, and let $A = (0,1]$ and $B = [1,2)$. Then $A \cup B = (0,2)$ and $\Int(A \cup B) = (0,2)$. However, $\Int(A) \cup \Int(B) = (0,1) \cup (1,2)$, so $\Int( A\cup B) \neq (\Int(A) \cup \Int(B))$. It is true, though, that $\Int(A) \cup \Int(B) \subseteq \Int(A \cup B)$. To see what, let $x \in \Int(A) \cup \Int(B)$. Then $x \in \Int(A)$ or $x \in \Int(B)$. Without loss of generality, assume that $x \in \Int(A)$. Then there is an open set $O$ in $A$ with $x \in O$. But $O \subseteq A$ implies $O \subseteq (A \cup B)$, so $x$ is also in $\Int(A \cup B)$. 

Finally, we will prove that $\Int(A)\cap \Int(B) = \Int(A \cap B)$. Let $x \in \Int(A) \cap \Int(B) $. Then $x \in \Int(A)$ and $x \in \Int(B)$. So there exist open sets $O_A$ in $A$ and $O_B$ in $B$ such that $x \in O_A$ and $x \in O_B$. From this we have that $O = O_A \cap O_B$ is an open set contained in $A \cap B$ and $x \in O$. Thus, $x \in \Int(A \cap B)$. So $\Int(A)\cap \Int(B) \subseteq  \Int(A \cap B)$.

Now suppose that $x \in \Int(A \cap B)$. Then there is an open set $O$ in $A \cap B$ with $x \in O$. But $O \subseteq A$ and $O \subseteq B$, so $x \in \Int(A)$ and $x \in \Int(B)$. It follows that $x \in \Int(A) \cap \Int(B)$. So $\Int(A \cap B) \subseteq  \Int(A)\cap \Int(B)$. These two inclusions show that  $\Int(A \cap B) = \Int(A)\cap \Int(B)$.

\ee

\end{comment}

\csection{The Interior of a Set in a Topological Space}

One might expect that the interior of a set is an open set, as it was in metric spaces. This is true, but we can say even more. In our preview activity we saw that in our examples that $\Int(A)$ was the largest open subset of $X$ contained in $A$. That this is always true is the subject of the next theorem. 

\begin{theorem} \label{thm:Interior} Let $(X,d)$ be a topological space, and let $A$ be a subset of $X$. Then interior of $A$ is the largest open subset of $X$ contained in $A$.  
\end{theorem}

\begin{proof} Let $X$ be a topological space, and let $A$ be a subset of $X$. We need to prove that $\Int(A)$ is an open set in $X$, and that $\Int(A)$ is the largest open subset of $X$ contained in $A$. First we demonstrate that $\Int(A)$ is an open set. Let $a \in \Int(A)$. Then $a$ is an interior point of $A$, so $A$ is a neighborhood of $a$. This implies that there exists an open set $O$ containing $a$ so that $O \subseteq A$. But $O$ is a neighborhood of each of its points, so every point in $O$ is an interior point of $A$. It follows that $O \subseteq \Int(A)$. Thus, $\Int(A)$ is a neighborhood of each of its points and, consequently, $\Int(A)$ is an open set. 

The proof that $\Int(A)$ is the largest open subset of $X$ contained in $A$ is left for the next activity.
\end{proof}

\begin{activity} Let $(X,d)$ be a topological space, and let $A$ be a subset of $X$. 
\ba
\item What will we have to show to prove that $\Int(A)$ is the largest open subset of $X$ contained in $A$?

\item Suppose that $O$ is an open subset of $X$ that is contained in $A$, and let $x \in O$. What does the fact that $O$ is open tell us? Then complete the proof that $O \subseteq \Int(A)$.

\ea

\end{activity}

\begin{comment}

\ActivitySolution

\ba
\item To prove that $\Int(A)$ is the largest open subset of $X$ contained in $A$ we need to prove that any open subset of $X$ that is contained in $A$ is a subset of $\Int(A)$.

\item Suppose that $O$ is an open subset of $X$ that is contained in $A$, and let $x \in O$. Since $O$ is a neighborhood of $x$, it follows that $A$ is a neighborhood of $x$ and $x \in \Int(A)$. It follows that $O \subseteq \Int(A)$ and $\Int(A)$ is the largest open subset of $X$ contained in $A$. 

\ea

\end{comment}

One consequence of Theorem \ref{thm:Interior} is the following.

\begin{corollary} A subset $O$ of a metric space $X$ is open if and only if $O = \Int(O)$. 
\end{corollary}
 